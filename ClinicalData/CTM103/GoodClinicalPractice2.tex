Good Clinical Practice ( GCP )
%==========================================%

\begin{itemize}
\item Good Clinical Practice (GCP) is a set of internationally recognised ethical and scientific quality requirements that must be observed for designing, conducting, recording and reporting clinical trials that involve the participation of human subjects. 

\item  Compliance with GCP provides assurance that the rights, safety and well being of trial subjects are protected, and that the results of the clinical trials are accurate and credible.  

\item  The Regulations require that all clinical trials covered by the provisions of the Regulations, including bioavailability and bioequivalence studies, shall be designed, conducted and reported in accordance with the principles of GCP. 

\item Good Clinical Practice (GCP) inspections are carried out in order to establish compliance with relevant legislation and guidelines. All inspections are carried out at “trial sites”.  

\item  A trial site can be defined as the location(s) where trial related activities are actually conducted.  Therefore an inspection can occur at such trial sites as the investigator site, sponsor company/CRO site, laboratory site and manufacturing site.  

\item A GCP inspection of a trial site may be performed on a routine basis or be scheduled in response to a request from another department or committee of the IMB where there is a specific cause for concern with respect to compliance with guidelines or regulations. 

\item A request to inspect a trial site may also be received from the Competent Authority of another EU member state or the Committee for Human Medicinal Products (CHMP).
\end{itemize}

%===============================================================%
