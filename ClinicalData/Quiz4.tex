%-------------------------------------------------------------------------%
Question 1
A case-control study of esophageal cancer was performed. Daily alcohol consumption was ascertained (H = high, L = low). The data was stratified by 2 age groups.

Low age group

H	L
Case	8	5
Control	52	164
High age group

H	L
Case	25	21
Control	29	128
Assuming a constant odds ratio across age-strata, test to see if the odds ratio is 1 and report a Pvalue

Your Answer		Score	Explanation
The Pvalue is modest (greater than .05), suggesting that the common odds ratio is different than one	Inorrect	0.00	
The Pvalue is small (less than .05), suggesting that the common odds ratio is equal to one			
The Pvalue is modest (greater than .05), suggesting that the common odds ratio is equal to one			
The Pvalue is small (less than .05), suggesting that the common odds ratio is different than one			
Total		0.00 / 1.00	
Question Explanation

We use the Mantel Haensel test. The R code is as follows.

\begin{framed}
\begin{verbatim}
dat <- array(c(8, 52, 5, 164, 25, 29, 21, 128), c(2, 2, 2))
dat
, , 1

     [,1] [,2]
[1,]    8    5
[2,]   52  164

, , 2

     [,1] [,2]
[1,]   25   21
[2,]   29  128
mantelhaen.test(dat, correct = FALSE)

    Mantel-Haenszel chi-squared test without continuity correction

data:  dat 
Mantel-Haenszel X-squared = 32.18, df = 1, p-value = 1.404e-08
alternative hypothesis: true common odds ratio is not equal to 1 
95 percent confidence interval:
 2.843 9.502 
sample estimates:
common odds ratio 
            5.197 
\end{verbatim}
\end{framed}
%-------------------------------------------------------------------------%
Question 2
The Biostatistics and Epidemilogy departments are running a 10K road race. There are three pairs of runners. In each case, a runner from Biostat was matched to a runner from Epi of the same age, gender and degree of running experience. The difference in each pairs times was taken and a signed rank test performed.

What is the smallest value that the two sided exact signed rank p-value could take?

Your Answer		Score	Explanation
0.125			
0.25			
0.75			
It can not be determined from the information given	Inorrect	0.00	
Total		0.00 / 1.00	
Question Explanation

## all possible collections of winners, 1 if biostat wins 0 if epi each
## column represent a possibility, there are 2^3 = 8 possibilities
winners <- t(expand.grid(c(1, 0), c(1, 0), c(1, 0)))
ranks <- 1:3
table(apply(winners, 2, function(x) sum(x * ranks)))/8

    0     1     2     3     4     5     6 
0.125 0.125 0.125 0.250 0.125 0.125 0.125 
The smallest value would occur with either 6 or 0 as the statistic, each occuring with probability 0.125. Twice that would be 0.25.
%-------------------------------------------------------------------------%
Question 3
Two computer based methods for diagnostic imaging are being studied. Ten images were processed with both methods, resulting in a + or − diagnosis for each method and image. The data are as follows

+	-
+	55	41
-	12	20
Where Method A is represeted with the two columns and Method B is represented on the rows. What is the P-value for a of the hypothesis that the marginal probability of a positive diagnosis is the same for the two methods? (Use the large sample version.)

\begin{itemize}
\item The P-value is non-significant, suggesting that the marginal probabilities do not differ.			
\item The P-value is highly significant, suggesting that the marginal probabilities do not differ.	Inorrect	0.00	
\item The P-value is highly significant, suggesting that the marginal probabilities do differ.			
\item The P-value is non-significant, suggesting that the marginal probabilities do differ.			
\end{itemize}
%Total		0.00 / 1.00	
%Question Explanation

We use the large sample version of McNemar's test
\begin{framed}
\begin{verbatim}
mcnemar.test(matrix(c(55, 12, 41, 20), 2), correct = FALSE)

    McNemar's Chi-squared test

data:  matrix(c(55, 12, 41, 20), 2) 
McNemar's chi-squared = 15.87, df = 1, p-value = 6.792e-05
\end{verbatim}
\end{framed}
%-------------------------------------------------------------------------%
Question 4
Two computer based methods for diagnostic imaging are being studied. Ten images were processed with both methods, resulting in a + or − diagnosis for each method and image. The data are as follows

+	-
+	55	41
-	12	20
Where Method A is represeted with the two columns and Method B is represented on the rows. What is the marginal odds ratio of a positive diagnosis comparing the two methods? (Put the odds associated with Method B in the numerator.)

Your Answer		Score	Explanation
The estimated odds of a positive diagnosis for Method B were about one half that of Method A.			
The estimated odds of a positive diagnosis for Method B were an order of magnitude than that of Method A.			
The estimated odds of a positive diagnosis for Method B were nearly 3 times that of Method A.	Correct	1.00	
The estimated odds of a positive diagnosis for Method B were about the same as that of Method A.			
Total		1.00 / 1.00	
Question Explanation
\begin{framed}
\begin{verbatim}
dat <- matrix(c(55, 12, 41, 20), 2)
n <- sum(dat)
pA <- sum(dat[, 1])/n
pB <- sum(dat[1, ])/n
oA <- pA/(1 - pA)
oB <- pB/(1 - pB)
or <- oB/oA
or
[1] 2.731
\end{verbatim}
\end{framed}
%-------------------------------------------------------------------------%
Question 5
A matched case control study was executed to study an airborn environmental toxicant's effect on lung cancer.The data are

Case and exposure status	count
Case exposed, control exposed	243
Case exposed, control unexposed	189
Case unexposed, control exposed	54
Case unexposed, control unexposed	153
What is the conditional odds ratio of odds of exposure for cases over odds of exposure for controls?

Your Answer		Score	Explanation
Around 3.5			
Around 0.30			
Around 2.4			
Around 4.2	Inorrect	0.00	
Total		0.00 / 1.00	
Question Explanation
\begin{framed}
\begin{verbatim}
 expCase <- 243 + 189
unexpCase <- 54 + 153
expCtrl <- 243 + 54
unexpCtrl <- 189 + 153

## condtional odds ratio
189/54
[1] 3.5
\end{verbatim}
\end{framed}
%-------------------------------------------------------------------------%
Question 6
A topical rash treatment was applied to a portion of a rash on n patients. A quantitative measure of redness was calculated for the treated and untreated regions of the rash. A sign of + was given when the treated area was less red (more improved) than the untreated area and a − sign when it was not. It is desired to know whether the treatment improves the rash.

How many possible values can the P-value for the sign test take?

Your Answer		Score	Explanation
There are n possible p-values			
There are n+1 possible p-values	Correct	1.00	
There are n (n + 1) / 2 possible p-values			
There is only one possible p-value			
Total		1.00 / 1.00	
Question Explanation

If there was one person, the test statistic could be 0 or 1 positive signs, thus the p-value has two possible values. For two people it could be 0, 1 and 2 for three possible values and so on.
Question 7
A topical rash treatment was applied to a portion of a rash on n patients. A quantitative measure of redness was calculated for the treated and untreated regions of the rash. A sign of + was given when the treated area was less red (more improved) than the untreated area and a − sign when it was not. It is desired to know whether the treatment improves the rash.

Consider a 5% one sided sign test for 5 subjects. What would be the power of the test if the probability that the treatment works is actually 80% instead of the 50% assumed under the null?

Your Answer		Score	Explanation
Around 95%			
Around 5%			
Around 60%	Inorrect	0.00	
Around 33%			
Total		0.00 / 1.00	
Question Explanation

By the relevant homework question, we will only reject for 5 subjects when all of them are +. The probability of such an occurrence under the alternative (where p=.8 rather than .5) is

0.8^5
TRUE [1] 0.3277
