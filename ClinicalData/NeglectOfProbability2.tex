he Neglect of Probability Bias: What is normal for the market?

The failures of large investment and insurance houses—Bear Sterns, Lehman Brothers, AIG, Freddie Mac, Fannie Mae—demonstrate that the models these companies used to manage risk were deeply flawed. The defense offered by many on Wall Street is that market events of the past year have been so extreme, so far from the norm, no one could have predicted or planned for such as financial catastrophe. Repeatedly financial and political leaders have compared the past year to events of the Great Depression. It appears that the risk models failed because market contractions of the current magnitude were not considered a possibility.

Extreme events by definition have low probabilities of occurrence. But low probability does not equal zero probability. Was it reasonable for the financial analysts to consider current events in the markets so far outside the norm that they would not occur? I keep hearing that current market conditions are not normal. But, in reflecting back over the past few decades I’m hard-pressed to think of any time that was “normal.”

During the mid-1990s stocks increased so fast that investors not making 20% per year felt left out of the boom. At the beginning the bull market in 1995 the Standard and Poors 500 index increased 34% in that year alone. From 1995 to 1999 it had returns in excess of 19% for each of those 5 years, more than doubling its overall value. Plenty of warnings were sounded that those market conditions were not normal and would not last. The crash at the end of the decade revealed that many companies simply made up numbers on earnings reports to drive the increase in stock prices. High-flying companies of the 1990s such as Enron, Global Crossing, Worldcom became synonyms for fraud and a number of their top executives are still in prison.

The recession of 1991-92 was so deep it cost President George H. W. Bush his job. Despite winning a popular war, he was unseated by Bill Clinton’s laser-like focus on economic problems. Everyone remembers “The economy, stupid” sign hung at his campaign headquarters.

In 1987 the stock market lost 25% of its value on a single trading day. At today’s valuation levels that would be the equivalent of a one-day drop of 2000-points.

During the mid-1980s certificate of deposits paid double-digit interest rates. I remember owning a one-year certificate of deposit that paid 10% annual interest. I also remember a life insurance salesman running a projection on the future value of a whole life policy that he was trying to sell me. Based on just small premiums (about $100 per month) he calculated an impressive future value of over a million dollars by the time I would retire in 40 years. Of course his calculation assumed an annual interest rate of 12% in perpetuity—a laughable projection given that annual rates on money market balances today are less than 1%.

The recession of 1981-82 resulted in unemployment greater than 10%, a rate we have yet to reach in the current recession.

I could extend this list of extreme economic events and conditions indefinitely back in time. But, the above reflection on events over the past 30 years shows that it is a fallacy to believe extreme events are outside of the “norm” and unlikely to happen. Instead it is apparent that extreme events are the norm. History is not going to stop and economic conditions, whether part of a boom or bust, never continue indefinitely.

But, I have noticed a tendency for financial planners and prognosticators to assume that economic conditions of the moment—whatever conditions are at that moment—will continue indefinitely. Much of the financial advice on buying, financing, and investing in homes over the past decade was all based on the assumption that prices in the housing market would only go up. This belief mirrors beliefs in the mid-1990s that stocks could only go up. The same thinking led my life insurance salesman in the mid-1980s to argue that interest rates on savings would be in the double digits forever.

The future is always uncertain and psychologists who study how people make decisions under uncertainty have identified a long list of “cognitive biases.” A bias refers to a repeated and predictable flaw in judgment that results in making less than optimal choices. For example, if you don’t know the future, optimal choice requires acting on the basis of the most probable outcomes. But the “neglect of probability bias" results in instances where people disregard probabilities when considering future outcomes.

Failure to use seat belts is an example of the neglect of probability bias. Car crashes are low probability events. You can drive for years, even decades and never be in a car crash. But, because the probability of crashing a car is not zero and consequence of even one crash potentially catastrophic, seat belts should be worn. The fact is car cashes occur with a rate predictable enough that auto insurance companies remain financially solvent. Evidently it is not that difficult to correctly price auto insurance.

Executives in banking and investing should consider devising something akin to a “financial seatbelt.” Rather than assume market crashes are too far outside the norm to worry about, they should accept the fact that market crashes have happened in the past and will happen in the future. They should have restraints in place ahead of time to limit the damage.

Joseph Ganem is a physicist and author of the award-winning The Two Headed Quarter: How to See Through Deceptive Numbers and Save Money on Everything You Buy
