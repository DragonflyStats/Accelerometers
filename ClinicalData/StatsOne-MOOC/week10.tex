\documentclass[12pt]{article}

%opening
\usepackage{framed}
\usepackage{amsmath}
\usepackage{amssymb}
\usepackage{graphicx}
\begin{document}


\section{Statistics One : Week 10 Quiz}
%--------------------------------------

Welcome to assignment 10! This week we are going to work on an example at the intersection of decision-making and global warming. 

\begin{itemize}
\item The simulated dataset includes a dependent variable, change, for a list of 27 countries. 

\item Change indicates whether these countries are willing to take action now against global warming, or if they would rather wait and see (1 = act now, 0 = wait and see). 

\item Predictors include: median age (age), education index (educ), gross domestic product (gdp), and CO2 emissions (co2).
\end{itemize}
%------------------------------------------------%
\subsection*{Question 1}
What is the median population age for the countries which voted to take action against global warming? (round to 2 decimal places)

\begin{framed}
\begin{verbatim}
describeBy(data$age, data$change=="1")
\end{verbatim}
\end{framed}
%------------------------------------------------%
\subsection*{Question 2}
Run a logistic regression including all predictor variables. Which predictors are significant in this model?

\begin{itemize}
\item educ and age			
\item educ, age, gdp	(NO)	
\item gdp, age, co2			
\item all of them		
\end{itemize}	


\begin{framed}
\begin{verbatim}
fit = glm(data$change ~ data$educ + data$age + 
 data$gdp + data$co2, family = binomial) 
summary(fit)
\end{verbatim}
\end{framed}
%-----------------------------------------------------------%
\subsection*{Question 3}

What does the negative value for the estimate of educ means?

\begin{itemize}
\item Countries with a lower education index score are more likely to chose to act now			
\item 
\item Countries with a higher education index score are more likely to chose to wait and see			
\item Educ and change are negatively correlated	(No)
\item All of the above
\end{itemize}

%-----------------------------------------------------------%
\subsection*{Question 4}
What is the confidence interval for educ, using profiled log-likelihood? (round to 2 decimal places, and give the lower bound first and the upper bound second, separated by a space)
	
\begin{framed}
\begin{verbatim}
confint(fit)
\end{verbatim}
\end{framed}
%-----------------------------------------------------------%
\subsection*{Question 5}
What is the confidence interval for age, using standard errors? (round to 2 decimal places, and give the lower bound first and the upper bound second, separated by a space)
\begin{framed}
\begin{verbatim}
confint.default(fit)
\end{verbatim}
\end{framed}

%-----------------------------------------------------------%
\subsection*{Question 6}
Compare the present model with a null model. What is the difference in deviance for the two models? (round to 2 decimal places)
\begin{framed}
\begin{verbatim}
with(fit, null.deviance - deviance)
\end{verbatim}
\end{framed}

%-----------------------------------------------------------%
\subsection*{Question 7}
How many degrees of freedom are there for the difference between the two models?
\begin{framed}
\begin{verbatim}
with(fit, df.null - df.residual)
\end{verbatim}
\end{framed}

%-----------------------------------------------------------%
\subsection*{Question 8}

Is the p-value for the difference between the two models significant?
\begin{framed}
\begin{verbatim}
with(fit, pchisq(null.deviance-deviance, 
  df.null-df.residual, lower.tail = FALSE))
\end{verbatim}
\end{framed}

%-----------------------------------------------------------%
\subsection*{Question 9}

Do chi-squared values differ significantly if you drop \textbf{\textit{educ}} as a predictor in the model?

\begin{framed}
\begin{verbatim}
library(aod)  
wald.test(b = coef(fit), Sigma = vcov(fit), Terms = 2)
\end{verbatim}
\end{framed}
%-----------------------------------------------------------%
\subsection*{Question 10}

What is the percentage of cases that can be classified correctly based on our model?

\begin{framed}
\begin{verbatim}
ClassLog(fit, data$change)
\end{verbatim}
\end{framed}
%-----------------------------------------------------------%
\end{document}
