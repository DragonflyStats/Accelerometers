Question 1
At a party a friend brags that they can tell the difference between cheap and expensive wine. You ask him to taste eight glasses wine for which you have randomized as 4 expensive and 4 cheap. Calculate a P-vlaue for the relevant exact hypothesis test using R's fisher.test function.

Actually Expensive	Actually Cheap
Guess Expensive	4	2
Guess Cheap	1	6
Your Answer		Score	Explanation
Around 0.5			
Around 0.4			
Around 0.1	Correct	1.00	
Around 0.6			
Around 0.2			
Around 0.3			
Around 0.7			
Around 0.8			
Around 0.9			
Total		1.00 / 1.00	
Question Explanation

fisher.test(matrix(c(4, 1, 2, 6), 2))

    Fisher's Exact Test for Count Data

data:  matrix(c(4, 1, 2, 6), 2) 
p-value = 0.1026
alternative hypothesis: true odds ratio is not equal to 1 
95 percent confidence interval:
   0.5365 682.2666 
sample estimates:
odds ratio 
     9.471 
Question 2
A drug treatment (T = treatment, P = Placebo) has been randomized to 5 patients in each arm. Consider testing an increase in side effects (S = side effects, N = none). The data are

T	P
S	3	1
N	2	4
Calculate Fisher's exact P-value for a one sided test of the relevant hypothesis.

Your Answer		Score	Explanation
Around 0.10			
Around 0.05			
Around 0.50			
Less than 0.01	Inorrect	0.00	
Around 0.25			
Total		0.00 / 1.00	
Question Explanation

fisher.test(matrix(c(3, 2, 1, 4), 2), alternative = "g")

    Fisher's Exact Test for Count Data

data:  matrix(c(3, 2, 1, 4), 2)
p-value = 0.2619
alternative hypothesis: true odds ratio is greater than 1
95 percent confidence interval:
 0.3152    Inf
sample estimates:
odds ratio 
     4.918 
{
    choose(4, 3) * choose(6, 2)/choose(10, 5) + choose(4, 4) * choose(6, 1)/choose(10, 
        5)
}
[1] 0.2619
Question 3
In a review of the author Jane Austen's work, scholars found the following relative frequencies of the words an'',that'' and “this''

an	that	this
0.53	0.35	0.12
An new story claimed to be Austen's was discovered with the word "an” 140 times, the word “that” 100 times and the word “this” 50 times. Are these counts consistent with Austen's traditional frequencies? Give a chi-squared P-value for the relevant test.

Your Answer		Score	Explanation
Around 0.05			
Around 0.50			
Around 0.25			
Greater than 0.75	Inorrect	0.00	
Total		0.00 / 1.00	
Question Explanation

This would be a goodness of fit test.

p <- c(0.53, 0.35, 0.12)
o <- c(150, 100, 50)
n <- sum(o)
e <- n * p
xs <- sum((o - e)^2/e)
pchisq(xs, lower.tail = FALSE, df = 2)
[1] 0.04523
chisq.test(o, p = p, correct = FALSE)

    Chi-squared test for given probabilities

data:  o
X-squared = 6.192, df = 2, p-value = 0.04523
Question 4
Two drugs, A and B , are being investigated in a randomized trial with the data are given below. Investigators would like to know if the Drug A and Drug B are equivalent in the terms of the distribution of side effects.

None	Nausea	Nauesea and Vomiting	Total
Drug A	80	15	5	100
Drug B	60	30	10	100
What are the expected cell counts under the null hypothesis?

Your Answer		Score	Explanation
33.3, 33.3 and 33.3 for each row			
140, 45 and 15 for each row			
70, 22.5, 7.5 for each row	Correct	1.00	
80, 15 and 5 for row 1 (Drug A) and 60, 30 and 10 for row 2 (Drug B)			
Total		1.00 / 1.00	
Question Explanation

p <- c(80 + 60, 15 + 30, 5 + 10)/200
p * 100
[1] 70.0 22.5  7.5
Question 5
100 people who took a pain medication for chronic pain were asked if they still had pain one hour after receiving the dose. They received the dose on two separate weeks. The results are given below with the result of the first week's dose on the rows and the second week's on the columns.

Yes	No
Yes	43	4
No	8	45
Does it appear that the self report impact of the medication is independent from one week to the next? (Test using the relevant chi-squared test.)

Your Answer		Score	Explanation
Yes, they do appear to be independent; we would fail to reject a chi squared test of independence.	Inorrect	0.00	
No, they do not appear independent; we would reject a chi squared test of independence.			
No, the do not appear to be independent; we would fail to reject a chi squared test of independence.			
Yes, they do appear to be independent; we would reject a chi squared test of independence.			
Total		0.00 / 1.00	
Question Explanation

Note, nearly all of the counts fall on the main diagonal, suggesting that the first week's result and the second are highly correlated.

dat <- matrix(c(43, 8, 4, 45), 2)
chisq.test(dat)

    Pearson's Chi-squared test with Yates' continuity correction

data:  dat 
X-squared = 55.16, df = 1, p-value = 1.112e-13
Question 6
A researcher is studying migration patterns. She collected the location of the current and previous homes for subjects who moved across regions (Northeast, Souteast and West). She recorded the following:

NE	SE	W
NE	-	267	255
SE	135	_	139
W	240	234	-
Here the row is current home region and the column is the region the person moved to.

Here the diagonals are not included since she only studied subjects who moved between regions. She would like to know if the probability of moving from region a to b is the same as the probability of moving from region b to a for all regions a and b.

What can be said about the expected cell counts?

Your Answer		Score	Explanation
The expected cell counts would be the average of a cell and its symmetrically opposite cell.			
The expected cell counts would be the same as that of a test of independence.	Inorrect	0.00	
The expected cell counts would be the same as that of a goodness of fit test.			
The expected cell counts could not be determined.			
Total		0.00 / 1.00	
Question Explanation

This is unlike any of the chi-squared tests covered in class. Let πij be the probability for cell i,j. Then the test is Ho:πij=πji. Then the estimate under the null hypothesis would be π^ij=(nij+nji)/(n×2) so that the expected cell counts would be (nij+nji)/2.

Question 7
Standard decks of playing cards have 13 cards of each suit (diamonds, clubs, hearts, spades) for a total of 52 cards. Someone gives you a 52 card deck. You draw a card, record its suit, replace the card, shuffle the deck and repeat that process 200 times, obtaining the following table

Diamonds	Clubs	Hearts	Spades
46	54	49	51
Does the distribution of suits appear to be standard?

Your Answer		Score	Explanation
We would fail to reject a chi-squared GOF test; we can not rule out that the deck is standard			
We would reject a chi-squared GOF test; we can rule out that the deck is standard			
We would reject a chi-squared GOF test; we can not rule out that the deck is standard	Inorrect	0.00	
We would fail to reject a chi-squared GOF test; we can rule out that the deck is standard			
Total		0.00 / 1.00	
Question Explanation

dat <- c(46, 54, 49, 51)
chisq.test(dat)

    Chi-squared test for given probabilities

data:  dat 
X-squared = 0.68, df = 3, p-value = 0.8779
Question 8
Consider the table below comparing the self reported job stress levels of three occupations. 100 people from each of three occupations were surveyed.

Job	High	Low
Exec	65	35
Prof	70	30
Lion Tamer	15	85
Researchers are interested in whether or not the perception of job stress differs by occupation. Using the relevant chi-sqared test. What can be said about the researcher's hypothesis?

Your Answer		Score	Explanation
We reject; the stress levels appear to differ between jobs	Correct	1.00	
We fail to reject; the stress levels do not appear to be different between jobs			
We reject; the stress levels do not appear to be different between jobs			
We fail to reject; the stress levels do appear to be different between jobs			
Total		1.00 / 1.00	
Question Explanation

dat <- matrix(c(65, 70, 15, 35, 30, 85), 3)
chisq.test(dat)

    Pearson's Chi-squared test

data:  dat 
X-squared = 74, df = 2, p-value < 2.2e-16a
