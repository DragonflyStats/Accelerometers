CTM201 Protocols


\subsection{AIMS, OBJECTIVES AND AUDIENCE}
Overall aim The aim of this module is to be able to produce the protocol for a trial. Material
in this module will build on the work of the core CT modules (or similar
material), but will go further into the steps to be taken for preparing the
protocol for a trial. For MSc CT students, it is recommended that this module is
taken in the same year that students attempt the last of written papers.
Working in small groups over a 4 week period (in either autumn or spring),
students will critically appraise a systematic review provided and start drafting
the background section for a trial protocol. Students can then use this as a basis
for developing their own draft protocol for assessment. Note that this will not
be the ‘final’ version of the protocol, but one that would be ready for
consultation with stakeholders.
\subsection{Intended learning}
outcomes
By the end of this module, students should be able to:
 critically appraise a systematic review in order to develop and refine a
research question to be addressed in a trial,
 understand the necessary components of a protocol for a trial,
 identify the relevant methodological and practical issues for the protocol,
 design the framework for an appropriate trial protocol, working in a group,
 appreciate the ‘added value’ of working in a group,
 extend this framework to produce the protocol for a trial in specific setting
(individual assessment).
