\subsection{Clinical Data Management}

% http://www.ncbi.nlm.nih.gov/pmc/articles/PMC3326906/
Clinical Data Management (CDM) is a critical phase in clinical research, which leads to 
generation of high-quality, reliable, and statistically sound data from clinical trials.

This helps to produce a drastic reduction in time from drug development to marketing. 
Team members of CDM are actively involved in all stages of clinical trial right from 
inception to completion. They should have adequate process knowledge that helps maintain 
the quality standards of CDM processes. 

Various procedures in CDM including Case Report Form (CRF) designing, CRF annotation, 
database designing, data-entry, data validation, discrepancy management, medical coding, 
data extraction, and database locking are assessed for quality at regular intervals during a trial. 

In the present scenario, there is an increased demand to improve the CDM standards to meet 
the regulatory requirements and stay ahead of the competition by means of faster commercialization 
of product. With the implementation of regulatory compliant data management tools, 
CDM team can meet these demands. 

Additionally, it is becoming mandatory for companies to submit the data electronically. 
CDM professionals should meet appropriate expectations and set standards for data quality 
and also have a drive to adapt to the rapidly changing technology. 

This article highlights the processes involved and provides the reader an overview of 
the tools  and standards adopted as well as the roles and responsibilities in CDM.


%--------------------------------------------------------------------------------------%

\section{Clinical Trial Definitions}

\subsection{Cross-sectional study}

In a cross-sectional study the data are obtained from a random sample of the population at one
point in time. This gives a snapshot of a population.

Example: Based on a single survey of a specific population or a random sample thereof, we
determine the proportion of individuals with heart disease at one point in time. This is referred to
as the prevalence of disease. We may also collect demographic and other information which will
allow us to break down prevalence broken by age, race, sex, socio-economic status, geographic,
etc.

\subsection{Longitudinal studies}

In a longitudinal study, subjects are followed over time and single or multiple measurements of
the variables of interest are obtained. Longitudinal epidemiological studies generally fall into
two categories; prospective i.e. moving forward in time or retrospective going backward in
time. We will focus on the case where a single measurement is taken.

\subsection{Prospective study} In a prospective study, a cohort of individuals are identified who are free
of a particular disease under study and data are collected on certain risk factors; i.e. smoking
status, drinking status, exposure to contaminants, age, sex, race, etc. These individuals are
then followed over some specified period of time to determine whether they get disease or not.
The relationships between the probability of getting disease during a certain time period (called
incidence of the disease) and the risk factors are then examined.

\end{document}
