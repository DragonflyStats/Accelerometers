Population health has been defined as "the health outcomes of a group of individuals, including the distribution of such outcomes within the group".[1] It is an approach to health that aims to improve the health of an entire human population. This concept does not refer to animal or plant populations. A priority considered important in achieving this aim is to reduce health inequities or disparities among different population groups due to, among other factors, the social determinants of health, SDOH. The SDOH include all the factors: social, environmental, cultural and physical the different populations are born into, grow up and function with throughout their lifetimes which potentially have a measurable impact on the health of human populations.[2] The Population Health concept represents a change in the focus from the individual-level, characteristic of most mainstream medicine. It also seeks to complement the classic efforts of public health agencies by addressing a broader range of factors shown to impact the health of different populations. The World Health Organization's Commission on Social Determinants of Health, reported in 2008, that the SDOH factors were responsible for the bulk of diseases and injuries and these were the major causes of health inequities in all countries.[3] In the US, SDOH were estimated to account for 70% of avoidable mortality.[4]
 
From a population health perspective, health has been defined not simply as a state free from disease but as "the capacity of people to adapt to, respond to, or control life's challenges and changes".[5] The World Health Organization (WHO) defined health in its broader sense in 1946 as "a state of complete physical, mental, and social well-being and not merely the absence of disease or infirmity."[6][7]
