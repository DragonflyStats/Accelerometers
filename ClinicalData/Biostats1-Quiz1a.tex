Question 1
You meet a person at the bus stop and strike up a conversation. In the conversation, the person gives the strange answer that at least one of his two children is a girl. Thinking on this, you decide to do a probability calculation. Assuming only that genders are iid with 50% probability each, what is the chance of a two child family having two girls given the information that at least one is a girl?
1/2
1/3
0
1/4
1
Question 2
A web site (www.medicine.ox.ac.uk/bandolier/band64/b64-7.html) for home pregnancy tests cites the following: "When the subjects using the test were women who collected and tested their own samples, the overall sensitivity was 75%. Specificity was also low, in the range 52% to 75%." Suppose a subject has a positive test. Assume the lower bound for the specificity. What number is closest to the multiplier of the pre-test odds of pregnancy to obtain the post-test odds of pregnancy given a positive test result?
2
2.5
0
1
1.5
.5
Question 3
A web site (www.medicine.ox.ac.uk/bandolier/band64/b64-7.html) for home pregnancy tests cites the following: "When the subjects using the test were women who collected and tested their own samples, the overall sensitivity was 75%. Specificity was also low, in the range 52% to 75%." Assume the lower value for the specificity. Suppose a subject has a positive test and that 30% of women taking pregnancy tests are actually pregnant. What number is closest to the probability of pregnancy given the positive test?
10%
30%
60%
80%
90%
40%
20%
50%
70%
Question 4
Let X1…XK be independent Poisson counts with means tiλ for some known value ti. Recall the Poison mass function with mean μ is μxe−μx! for x=0,1,…. What is the maximum likelihood estimate for λ?
(∑Kk=1xk)/n
(∑Kk=1tk)/(∑Kk=1xk)
(∑Kk=1xk)/(∑Kk=1tk)
(∑Kk=1xk)×(∑Kk=1tk)
Question 5
Suppose that a person is flipping a biased coin with success probability p. She flips the coin 10 times yielding 1 head. Consider two possibilities: 1) the person planned on flipping the coin ten times and got one head, 2) the person planned to flip the coin until the first head and it took ten times. What can you say about the likelihood in these two circumstances?
The likelihoods are different (up to constants of proportionality) depending on which case is true.
The likelihood associated with p is identical (up to constants of proportionality) in either case.
You can't calculate a likelihood in setting 2.
You can't calculate a likelihood in setting 1.
Question 6
Let X be a uniform random variable with support of length 1, but where we don't know where it starts. So that the density is f(x)=1 for x∈(θ,θ+1) and 0 otherwise. We observe a random variable from this distribution, say x1. What does the likelihood look like?
A diagonal line from x1 to x1+1.
A parabola.
A horizontal line between x1 and x1−1.
A point at x1.
Question 7
Suppose that diastolic blood pressures (DBPs) for men aged 35-44 are normally distributed with a mean of 80 (mm Hg) and a standard deviation of 10. What is the probability that a random 35-44 year old has a DBP less than 70?
About 16%
About 50%
About 1%
About 68%
About 2.5%
About 34%
Question 8
Brain volume for adult women is normally distributed with a mean of about 1,100 cc for women with a standard deviation of 75 cc. About what brain volume represents the 95th percentile?
1100 cc
1250 cc
1200 cc
1220 cc
