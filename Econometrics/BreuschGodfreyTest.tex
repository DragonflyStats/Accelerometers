\secton*{Breusch–Godfrey Test}
\begin{itemize}
\item In statistics, the Breusch–Godfrey test, named after Trevor S. Breusch and Leslie G. Godfrey, is used to assess the validity of some of the modelling assumptions inherent in applying regression-like models to observed data series. 
\item In particular, it tests for the presence of serial correlation that has not been included in a proposed model structure and which, if present, would mean that incorrect conclusions would be drawn from other tests, or that sub-optimal estimates of model parameters are obtained if it is not taken into account. 
\item The regression models to which the test can be applied include cases where lagged values of the dependent variables are used as independent variables in the model's representation for later observations. 
\item This type of structure is common in econometric models.
\item Because the test is based on the idea of Lagrange multiplier testing, it is sometimes referred to as LM test for serial correlation.
\item A similar assessment can be also carried out with the Durbin–Watson test and the Ljung–Box test.
\end{itemize}

\end{document}
