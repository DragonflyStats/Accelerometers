Stochastic Regressors and Measurement Errors

Linear regression assumed that the explanatory variables, the regressors have no random values, and that their values in the sample are fixed. 
That is to say, they are non-stochastic.

In stochastic regression, the values are assumed to be drawn randomly from defined populations. This is a
much more realistic framework for regressions wtih cross-sestional data.

\begin{enumerate}
\item The model is linear and correctly specified
\item The values of the stochastic regressors are drawn randomly from fixed populations.
\item There does not exist an exact linear relationship between the regressors.
\item The disturbance term has zero expectation 
\[ E(U_i)=0\]
\item The disturance term is homoscedascistic
\[ \sigma^2(u_i) = \sigma^2(u_j) = \ldots= \sigma^2(u)\]
\item The values of the disturbance terms have independent distributions.

\item The disturbance term is distributed independently of the regressors.
\item The disturbance term has zero conditional expectation.

\end{enumerate}

