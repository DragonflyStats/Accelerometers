\documentclass[]{article}

%opening
\title{}
\author{}

\begin{document}
In general, the following recommendations hold:
\begin{description}
\item[Recommendation 1:] Carefully choose the set of key variables using knowledge
of both subject matter experts and disclosure control experts. As already men-
tioned, the key variables are those variables for which an intruder may possible
have data/ information, e.g. age and region from persons or turnover of enter-
prises. Which external data are available containing information on key variables
is usually known by subject matter specialist.
\item[Recommendation 2:] Always perform a frequency and risk estimation to evaluate
how many observations have a high risk of disclosure given the selection of key
variables.
\item[Recommendation 3:] Apply recoding to reduce uniqueness given the set of cat-
egorical key variables. This approach should be done in an exploratory manner.
Receding on a variable, however, should also be based on expert knowledge to
combine appropriate categories. Alternatively, swapping procedures may be ap-
plied on categorical key variables so that data intruders cannot be certain if an
observation has or has not been perturbed.
\item[Recommendation 4:] If recoding is applied, apply local suppression to achieve
k-anonymity. In practice, parameter It" is often set to 3.
\item[Recommendation 5:] Apply micro-aggregation to continuously scaled key vari-
ables. This automatically provides k-anonymity for these variables.
\item[Recommendation 6:] Quantify the data utility not only by using typical esti-
mates such as quantiles or correlations, but also by using the most important
data-specific benchmarking indicators
\end{description}
\end{document}