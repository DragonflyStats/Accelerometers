%% - http://link.springer.com/article/10.1007%2Fs10618-007-0078-6#page-1
\documentclass[SDCmaster.tex]{subfiles}
\begin{document}
	
	\begin{frame}
		\frametitle{Statistical Disclosure Control - Concepts}
		\Large
		\begin{itemize}
\item Suda 2 is a recursive algorithm for finding \textbf{Minimal Sample Uniques}.
\item The algorithm generates all
possible variable subsets of defined categorical key variables and scans them for unique patterns in
the subsets of variables.
\item The lower the amount of variables needed to receive uniqueness, the higher
the risk of the corresponding observation.
\end{itemize}
\end{frame}
%=========================================================================================== %
\begin{frame}
	\frametitle{Statistical Disclosure Control - Concepts}
	\Large
	\begin{itemize}
\item A new algorithm, SUDA2, is presented which finds minimally unique itemsets i.e., minimal itemsets of frequency one. 
\item These itemsets, referred to as Minimal Sample Uniques (MSUs), are important for statistical agencies who wish to estimate the risk of disclosure of their datasets. 
\item SUDA2 is a recursive algorithm which uses new observations about the properties of MSUs to prune and traverse the search space.
\item Experimental comparisons with previous work demonstrate that SUDA2 is several orders of magnitude faster, enabling datasets of significantly more columns to be addressed. 
\item The ability of SUDA2 to identify the boundaries of the search space for MSUs is clearly demonstrated.
\end{itemize}
\end{frame}
\end{document}