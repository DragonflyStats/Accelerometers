http://neon.vb.cbs.nl/casc/SDC_Handbook.pdf

%=========================================================================================%
% SKINNER -LSE

Statistical disclosure control refers to the methodology used in the design of the statistical 
outputs from a survey for protecting the confidentiality of respondents’ answers. The 
threat to confidentiality is assumed to come from a hypothetical intruder who has access 
to these outputs and seeks to use them to disclose information about a survey respondent. 

%--------------------%

One key concern relates to identity disclosure, which would occur if the intruder were 
able to link a known individual (or other unit) to an element of the output. Another main 
concern relates to attribute disclosure, which would occur if the intruder could determine 
the value of some survey variable for an identified individual (or other unit) using the 
statistical output. Measures of the probability of disclosure are called \textbf{disclosure risk}. 

If 
this level of risk is deemed unacceptable then it may be necessary to apply a method of 
statistical disclosure control to the output. The choice of which method and how much 
protection to apply depends not just on the impact on disclosure risk but also on the 
impact on the utility of the output to users. This paper provides a review of statistical 
disclosure control methodology for two main types of survey output: (i) tables of 
estimates of population parameters and (ii) microdata, often released as a rectangular file 
of variables by analysis units. For each of these types of output, the definition and 
estimation of disclosure risk is discussed as well as methods for statistical disclosure 
control. 

1. Introduction 
1.1. The problem of statistical disclosure control
Survey respondents are usually provided with an assurance that their responses will 
be treated confidentially. These assurances may relate to the way their responses will be 
handled within the agency conducted the survey or they may relate to the nature of the 
statistical outputs of the survey as, for example, in the ‘confidentiality guarantee’ in the 
United Kingdom (UK) National Statistics Code of Practice (National Statistics, 2004, 
p.7) that ‘no statistics will be produced that are likely to identify an individual’. This 
paper is concerned with methods for ensuring that the latter kinds of assurances are met. 
Thus, in the context of this paper, statistical disclosure control (SDC) refers to the 
methodology employed, in the design of the statistical outputs from the survey, for 
protecting the confidentiality of respondents’ answers. Methods relating to the first kind 
of assurance, for example computer security and staff protocols for the management of 
data within the survey agency, fall outside the scope of this paper. 
There are various kinds of statistical outputs from surveys. The most traditional are 
tables of descriptive estimates, such as totals, means and proportions. The release of such
estimates from surveys of households and individuals have typically not been considered 
to represent a major threat to confidentiality, in particular because of the protection 
provided by sampling. Tabular outputs from the kinds of establishment surveys 
conducted by government have, however, long been deemed risky, especially because of 
the threat of disclosure of information about large businesses in cells of tables which are 
sampled with a 100% sampling fraction. SDC methods for such tables have a long history 
and will be outlined in Section 2. 
 While the traditional model of delivering all the estimates from a survey in a single 
report continues to meet certain needs, there has been increasing demand for more 
flexible survey outputs, often for multiple users, where the set of population parameters 
of interest is not pre-specified. There are several reasons why it may not be possible to 
pre-specify all the parameters. Data analysis is an iterative process and what analyses are 
of most interest may only become clear after initial exploratory analyses of the data. 
Moreover, given the considerable expense of running surveys, it is natural for many 
commissioners of surveys to seek to facilitate the use of the data by multiple users. But it 
is usually impossible to pre-specify all possible users and their needs in advance. A 
natural way to provide flexible outputs from a survey to address such needs is to make 
the survey microdata available so that users can carry out the statistical analyses that 
interest them. 
The release of such microdata raises serious confidentiality protection issues, 
however. Of course, statistical analyses of survey data do not require that the identities of 
the survey units are known. Names, addresses and contact information for individuals or 
establishment can be stripped from the data to form an anonymised microdata file. The 
problem, however, is that such basic anonymisation is often insufficient to protect 
confidentiality, and it is necessary therefore to employ one of a range of alternative 
approaches to SDC and this will be discussed further in Section 3. 
%====================================================================================================%


4. Conclusion 
The development of SDC methodology continues to be stimulated by a wide range of 
practical challenges and by ongoing innovations in the ways that survey data are used, 
with no signs of diminishing concerns about confidentiality. There has been a tendency 
for some SDC methods to be developed in somewhat ad hoc way to address specific 
problems and one aim of this paper has been to draw out some principles and general 
approaches which can guide a more unified methodological development. Statistical 
modelling has provided one important framework for this purpose. Other fields with the 
potential to influence the systematic development of SDC methodology in the future 
include data mining, in particular methods related to record linkage, and approaches to 
privacy protection in computer science and database technology. 
%===================================================================================================%
