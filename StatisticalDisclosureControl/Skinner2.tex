


and de Waal (2001, Sect. 9.2). A simple method is rounding, where the modified cell
values are multiples of a given base integer (Willenborg and de Waal, 2001, Ch. 9). This
method is more commonly applied to frequency tables derived from 100\% data but can
also be applied to tables of estimated totals from surveys, where the base integer may be
chosen according to the magnitudes of the estimated totals. Instead of replacing the cell
values by single safe values, it is also possible to replace the values by intervals, defined
by lower and upper bounds (Salazar, 2003; Giessing and Dittrich, 2006). The method of
controlled tabular adjustment (Cox et al.,2004) determines modified cell values within
such bounds so that the table remains additive and certain safety and statistical properties
are met.

Pre-tabular microdata modification: instead of modifying the cell values, the
underlying microdata may be perturbed, e.g. by adding noise, and then the table formed
from the perturbed microdata (Evans, et al. 1998; Massell et al., 2006).

%------------------------------------------------------------%
The statistical output from a survey will typically include many tables. Although the
above methods may be applied separately to each table, such an approach takes no
account of the possible additional disclosure risks arising from the combination of
information from different tables, in particular, from common margins. To protect against
such additional risks raises new considerations for SDC. Moreover, the set of tables
constituting the statistical output is not necessarily fixed, as in a traditional survey report.

With developments in online dissemination, there is increasing demand for the generation
of tables which can respond in a more flexible way to the needs of users. This implies the
need to consider SDC methods which not only protect each table separately as above, but
also protect against the risk arising from alternative possible sequences of released tables
(see e.g. Dobra et al, 2003).

%----------------------------------------------------------%
\newpage
\section*{3. Microdata}

\subsection*{P3.1. Assessing disclosure risk}

We suppose the agency is considering releasing to researchers an anonymised
microdata file, where the records of the file correspond to the basic analysis units and
each record contains a series of survey variables. The record may also include identifiers
for higher level analysis units, e.g. household identifiers where the basic units are
individuals, as well as information required for survey analysis such as survey weights
and primary sampling unit (PSU) identifiers.

We suppose that the threat of concem is that an intmder may link a record in the file
to some external data source of known units using some variables, which are included in
both the microdata file and the external source. These variables are often called key
variables or identifying variables. 

There are various ways of defining disclosure risk in
this setting. See e.g. Paass (1988) and Duncan and Lambert (1989). A common approach,
often motivated by the nature of the confidentiality pledge, is to consider a form of
identification risk (Bethlehem et al., 1990; Reiter, 2005), concemed with the possibility
that the intmder will be able to determine a correct link between a microdata record and a
known unit. 

This definition of risk will only be appropriate if the records in the microdata
can meaningfully be said to be associated with units in the population. When microdata is
subject to some forms of SDC, this may not be the case (e.g. if the released records are
obtained by combining original records) and in this case it may be more appropriate to
consider some definition of predictive disclosure (e.g. Fuller, 1993) although we do not
pursue this further here.

A number of approaches to the assessment of identification risk are possible, but all
depend importantly upon assumptions about the nature of the key variables. One
approach is to conduct an empirical experiment, matching the proposed microdata against
another data source, which is treated as a surrogate for the data source held by the
intruder. Having made assumptions about the key variables, the agency can use record
linkage methods, which it is plausible would be available to an intruder, to match units
between the two datasets. 

Risk might then be measured in terms of the number of units
for which matches are achieved together with a measure of the match quality (in terms of
the proportions of false positives and negatives). Such an experiment therefore requires
that the agency has information which enables it to establish precisely which units are in
common between the two sources and which are not.

The key challenge in this approach is how to construct a realistic surrogate intruder
dataset, for which there is some overlap of units with the microdata and the nature of this
overlap is known. On some occasions a suitable alternative data source may be available.
Blien et al. (1992) provide one example of a data source listing people in certain
occupations. 

Another possibility might be a different survey undertaken by the agency,
although agencies often control samples to avoid such overlap. Even if there is overlap,
say with a census, determining precisely which units are in common and which are not
may be resource intensive. Thus, this approach is unlikely to be suitable for routine use.
In the absence of another dataset, the agency may consider a re-identification
experiment, in which the microdata file is matched against itself in a similar way,
possibly after the application of some SDC method (Winkler, 2004). This approach has
the advantage that it is not model-dependent, but it is possible that the re-identification
risk is over-estimated if the disclosure protection effects of sampling and measurement
error are not allowed for in a realistic way.

In the remainder of section 3, we consider a third approach, which again only
requires data from the microdata file, but makes theoretical assumptions, especially of a
modelling kind, in order to estimate identification risk. As for the re~identification
experiment, this approach must make assumptions about how the key variables are
measured in the microdata and by the intruder on known units using external information.

A simplifying but ‘worst case’ assumption is that the key variables are recorded in
identical ways in the microdata and externally. We refer to this as the n0 measurement
error assumption, since measurement error in either of the data sources may be expected
to invalidate this assumption. 

If at least one of the key variables is continuous and the no
measurement error assumption is made then an intruder who observes an exact match
between the values of the key variables in the microdata and on the known units could
conclude with probability one that the match is correct, in other words the identification
risk would be one. If at least one of the key variables is continuous and it is supposed that
measurement error may occur then the risk will generally be below one. Moreover, an
exact matching approach is not obviously sensible and a broader class of methods of
record linkage might be considered. See Fuller (1993) for the assessment of disclosure
risk under some measurement error model assumptions.

In practice, variables are rarely recorded in a continuous way in social survey
microdata. For example, age would rarely be coded with more detail than one year bands.

%-----------------------------------------------------------------------------------------------%

And from now on we restrict attention to the case of categorical key variables. For
simplicity, we restrict attention to the case of exact matching, although more general
record linkage methods could be employed. We focus on a microdata file, where the only
SDC methods which have been applied are recoding of key variables or random
(sub)sampling. We comment briefly on the impact of other SDC methods on risk in
section 3.4.
3.2 File-level measures of identification risk
We consider a finite population U of N units (which will typically be individuals)
and suppose the microdata file consists of records for a sample s c U of size n § N . We
assume that the possibility of statistical disclosure arises if an intruder gains access to the
microdata and attempts to match a microdata record to external information on a known
unit using the values of m categorical key variables X ,,...,X M . (Note that s and
X1,..., Xm are defined after the application of (sub)sampling or recoding respectively as
SDC methods to the original microdata file.)

%==============================================%
Let the variable formed by cross-classifying X1,...,Xm be denoted X , with values
denoted k = l,...,K , where K is the number of categories or key values of X . Each of
these key values corresponds to a possible combination of categories of the key variables.
Under the no measurement error assumption, identity disclosure is of particular concern
if a record is unique in the population with respect to the key variables. A record with key
value k is said to be population unique if F, =1, where F, denotes the number of units
in U with key value k . If an intruder observes a match with a record with key value k,
knows that the record is population unique and can make the no measurement error
assumption then the intruder can infer that the match is correct.

As a simple measure of disclosure risk, we might therefore consider taking some
summary of the extent of population uniqueness. In survey sampling it is usual to define
parameters of interest at the population level and this might lead us to define our measure
as the population proportion NI / N , where N, =2, I (F, = r) is the population
frequencies of frequencies, r =1, 2,.... From a disclosure risk perspective, however, we
are interested in the risk for a specific microdata file it is natural to allow the risk measure
to be sample dependent. Thus, we might expect the risk to be higher if a sample is
selected with a high proportion of unusual identifiable units than for a sample where this
proportion is lower. Thus, a more natural file-level measure is the proportion of
population uniques in the sample. Let the sample counterpart of F, be denoted by fk
then this measure can be expressed as:

Pr(PU)=Zk1(f,,=l,Fk=l)/n. (2)

It could be argued, however that the denominator of this proportion should be made even
smaller, since the only records which might possibly be population unique are ones that
are sample unique (since fk 3 Fk), i.e. have a key value k such that fk =1 . Thus a more
conservative measure would be to take:

\[Pr(PU | su) =Z,1(f, = 1, F, =1)/nl, \]
where n) is the number of sample uniques and, more generally, n,_ = 2, I ( fk : r) is the
sample frequencies of frequencies. For further consideration of the proportion of sample
uniques that are population unique, see Fienberg and Ma.kov (1998) and Samuels (1998).
It may be argued (e.g. Skinner and Elliot, 2002) that these measures may be over-
optimistic, since they only capture the risk arising from population uniques and not from
other records with E, 2 2. If an intruder observes a match on a key value with frequency
E, then (subject to the no measurement error assumption) the probability that the match
is correct is 1/ F, under the exchangeability assumption that the intruder is equally likely
to have selected any of the F, units in the population. 

An alternative measure of risk is
then obtained by extending this notion of probability of correct match across different
key values. Again, on worst case grounds, it is natural to restrict attention to sample
uniques. One measure arises from supposing that the intmder starts with the microdata, is
equally likely to select any sample unique and then matches this sample unique to the
population. The probability that the resulting match is correct is then the simple average
of 1/ E across sample uniques:
9.t=[Zt1<fi<=1>/Fkl/"1 (4)

Another measure is
61/:Z;<I(_f/(=1)/2/<I:I<I(.fi.:1)a (5)

which is the probability of a correct match under a scenario where the intruder searches at
random across the population and finds a match with a sample unique.
All the above four measures are functions of both the fk and the F, . The agency
conducting the survey will be able to determine the sample quantities fk from the
microdata but the population quantities Fk will generally be unknown. It is therefore of
interest to be able to make inference about the measures from sample data.
Skinner and Elliot (2002) show that, under Bernoulli sampling with inclusion
probability 71', a simple design-unbiased estimator of 0U is 63,, =n,/[n1+2(7f'—l)n2].
They also provide a design consistent estimator for the asymptotic variance of éu —6U .
Skinner and Carter (2003) show that a design-consistent estimator of 6” for an arbitrary
complex design is éu =11,/[nl+2(7T2" —l)n2], where 772" is the mean of the inverse
inclusion probabilities 7;?‘ for units iwith key values for which fk =2. They also
provide a design-consistent estimator of the asymptotic variance of §L,—6U under
Poisson sampling.

Such simple design-based inference does not seem to be possible for the other three
measures in (2) - (4). Assuming a symmetric design such as Bernoulli sampling, we
ll



might suppose that n,  represent sufficient statistics and seek design-based moment-
based estimators of the measures by solving the equations:
E(n,.)=Z1v,1>,,, r=l,Z,...
where the coefficients P” are known for sampling schemes such as simple random
sampling or Bernoulli sampling (Goodman, 1949). The solution of these equations for N,
with E(n,_) replaced by nr gives unbiased estimators of K and N, under apparently
weak conditions (Goodman, 1949). Unfortunately, Goodman found that the estimator of
K can be ‘very unreasonable’ and the same appears to be so for the corresponding
estimator of NI. Bunge and Fitzpatrick (1993) review approaches to estimating K and
discuss these difficulties. Zayatz (1991) and Greenberg and Zayatz (1992) propose an
altemative ‘nonparametric’ estimator of N] but this appears to be subject to serious
upward bias for small sampling fractions (Chen and Keller-McNulty, 1998).
One way of addressing these estimation difficulties is by making stronger
modelling assumptions, in particular by assuming that the E are independently
distributed as:
Ft l/it ~ P0(1i) (6)
where the /1, are independently and identically distributed, i.e. that the Fk follow a
compound Poisson distribution. A tractable choice for the distribution of /1, is the
gamma distribution (Bethlehem et al., 1990) although it does not appear to fit well in
some real data applications (e.g. Skinner et al., 1994; Chen and Keller-McNulty, 1998).

A much better fit is provided by the log-normal (Skinner and Holmes, 1993). Samuels
(1998) discussed estimation of Pr(PU ISU) based on a Poisson-Dirichlet model. A
general conclusion seems to be that results can be somewhat sensitive to the choice of
model, especially as the sampling fraction decreases, and that 49” can be more robustly
estimated than the other three measures.

\subsection*{3.3 Record-level measures of identification risk}
A concern with file-level measures is that the principles governing confidentiality
protection often seek to avoid the identification of any individual, that is require the risk
to be below a threshold for each record, and such aims may not adequately be addressed
by aggregate measures of the form (2) - (5). To address this concem, it is more natural to
consider record level measures, i.e. measures which may take different values for each
microdata record. Such measures may help identify those parts of the sample where risk
is high and more protection is needed and may be aggregated to a file level measure in
different ways if desired (Lambert 1993). While record level measures may provide
greater flexibility and insight when assessing whether specified forms of microdata
output are ‘clisclosive’, they are potentially more difficult to estimate than file level
measures.
A number of approaches have been proposed for the estimation of record level
measures. For continuous key variables, Fuller (1993) shows how to assess the record
l2



level probability of identification in the presence of added noise, under normality
assumptions. See also Paass (1988) and Duncan and Lambert (1989). We now consider
related methods for categorical variables, following Skinner and Holmes (1998) and
Elamir and Skinner (2006).

Consider a microdata record with key value X . Suppose the record is sample unique,
i.e. with a key value k for which fk =1 , since such records may be expected to be most
risky. Suppose the intruder observes an exact match between this record and a known
unit in the population. We make the no measurement error assumption so that there will
be F, units in the population which potentially match the record. We also assume no
response knowledge (see section 2.1). The probability that this observed match is correct
is
Pr(c0rrect match I exact match, X = k, Fk) =1/F, (7)
where the probability distribution is with respect to the design under a symmetric
sampling scheme, such as simple random sampling or Bemoulli sampling. (Alternatively,
it could be with respect to a stochastic mechanism employed by the intruder, which
selects any of the F, units with equal probability). This probability is conditional on the
key value k and on Fk .
%==============================================%
In practice, we only observe the sample frequencies fk and not the Fk. We therefore
integrate out over the uncertainty about 1*‘, and write the measure as
Pr(c0rrect match I exact match,X = k,fk) = E(l/F, lk, fk = l) (8)

This expectation is with respect to both the sampling scheme and a model generating
the Fk, such as the compound Poisson model in (6). An altemative measure, focusing on
the risk from population uniqueness, is:
\[ Pr(fi, =l|k,fk =1) (9)\]
The expressions in (8) and (9) may be generalized for any record in the microdata with
fk >1. A difference between the probabilities in (8) and (9) and those in the previous
section is that here we condition on the record’s key value X = k. Thus, although we
might assume Fk ll, ~ P0(/1,), as in (6), we should like to condition on the particular
key value k when considering the distribution of /1, . Otherwise, if the 2., are identically
distributed as in the previous section then we would obtain the same measure of risk for
all (sample unique) records. A natural model is a log-linear model:
1°g(/1k)= Zkfl (10)
wherezk is a vector of indicator variables representing the main effects and the
interactions between the key variables Xl,...,Xm and fl is a vector of unknown
parameters.
13

