\documentclass{beamer}

\usepackage{amsmath}
\usepackage{amssymb}

\begin{document}


%============================================== %
\begin{frame}
\frametitle{Recommendations (Templ et al)}



	Templ et al make the following recommendations
	\begin{description}
		\item[Recommendation 1:] Carefully choose the set of key variables using knowledge
		of both subject matter experts and disclosure control experts. \\ The key variables are those variables for which an intruder may possible
		have data/ information, e.g. age and region from persons or turnover of enterprises. \\ Which external data are available containing information on key variables
		is usually known by subject matter specialist?

\end{description}
\end{frame}
%============================================== %
\begin{frame}
	\frametitle{Recommendations (Templ et al)}
	\begin{description}
		\item[Recommendation 2:] Always perform a frequency and risk estimation to evaluate
		how many observations have a high risk of disclosure given the selection of key
		variables.
\end{description}
\end{frame}
%============================================== %
\begin{frame}
	\frametitle{Recommendations (Templ et al)}
	\begin{description}
		\item[Recommendation 3:] Apply recoding to reduce uniqueness given the set of categorical key variables.\\ This approach should be done in an exploratory manner.
		Recoding on a variable, however, should be based on expert knowledge to
		combine appropriate categories. \\ Alternatively, swapping procedures may be applied on categorical key variables so that data intruders cannot be certain if an
		observation has or has not been perturbed.
\end{description}
\end{frame}
%============================================== %
\begin{frame}
	\frametitle{Recommendations (Templ et al)}
	\begin{description}
		\item[Recommendation 4:] If recoding is applied, apply local suppression to achieve
		k-anonymity. In practice, parameter $k$ is often set to 3.
		\item[Recommendation 5:] Apply micro-aggregation to continuously scaled key variables. This automatically provides k-anonymity for these variables.
		\item[Recommendation 6:] Quantify the data utility not only by using typical estimates such as quantiles or correlations, but also by using the most important
		data-specfic benchmarking indicators
	\end{description}
\end{frame}
\end{document}
