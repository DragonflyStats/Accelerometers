\documentclass{beamer}

\usepackage{amsmath}
\usepackage{amssymb}

\begin{document}
	\begin{frame}
		\frametitle{sdcMicro}
		\begin{itemize}
			\item The \texttt{R} package \textbf{sdcMicro} is a well-known collection of microdata protection methods developed by \textbf{Statistics Austria}, which is already in use in several national statistics offices. 
			\item \textbf{sdcMicro} has become one of the standard tools for \textbf{microdata protection} during the last five years.
			\item The IHSN is supporting the further development of sdcMicro and has partnered with its developers to perform the following tasks (next slide).
		\end{itemize}
	\end{frame}
\begin{frame}
	\begin{itemize}
		\item Version 4.4.0 of the sdcMicro package is available on the Comprehensive R Archive Network (CRAN).
		\item Existing guidelines and a user guide for \textbf{sdcMicro} are being updated. 
		\item A specific tutorial is being developed to show how to implement these concepts and algorithms on real datasets (see 
		Vignettes)
	\end{itemize}
	
	
\end{frame}
	%======================================================= % 
\begin{frame}
\frametitle{\texttt{R}-Package sdcMicro and sdcMicroGUI}

\begin{itemize}
\item SDC methods introduced in this guideline can be implemented by the R-Package
	sdcMicro. 
\item Users who are not familiar with the native R command line interface
	can use sdcMicroGUI, an easy-to-use and interactive application. 
	%For details, see Tcnipl ct al. [2Ol'lb, 2013].
\end{itemize}
\end{frame}
%=========================================================== %
%\section*{1.3. Remarks on SDC Methods}
\begin{frame}
\frametitle{SDC Methods} 
\textbf{Mathematical Foundation of SDC}
\begin{itemize}
\item In general, SDC methods borrow techniques from other fields. 
\item For instance, multivariate (robust) statistics are used to modify or simulate continuous variables and
to quantify information loss. 

\item Distribution-fitting methods are used to quantify
disclosure risks.
\item Statistical modeling methods form the basis of perturbation algorithms, to simulate synthetic data, to quantify risk and information loss. 
\item Linear
programming is used to modify data but minimize the impact on data quality.
\end{itemize}
\end{frame}
%==================================================================== %
\begin{frame}
	\frametitle{SDC Methods} 
	\begin{itemize}
		\item
Problems and challenges arise from large datasets and the need for efficient algorithms and implementations.\\ (\textbf{Remark} 1000 variables+)
\item Another layer of complexity is produced by complex
structures of hierarchical, multidimensional data sampled with complex survey designs. \\ (\textbf{Remark} Rooted Questions )
\item 
Missing values are a challenge, especially for computation time issues; structural Zeros (values that are by definition Zero) also have impact on the application
of SDC methods. 
\end{itemize}
\end{frame}
%================================================================================ %
\begin{frame}
	\frametitle{SDC Methods} 
Furthermore, the compositional nature of many components
should always be considered, but adds even more complexity.\\ \bigskip
SDC techniques can be divided into three broad topics:
\begin{itemize}
\item Measuring disclosure risk % (see Section 2)
\item Methods to anonymize micro-data % (see Section 3)
\item Comparing original and modified data (information loss) % (see Section /l)
\end{itemize}
\end{frame}
%============================================================================== %
\end{document}