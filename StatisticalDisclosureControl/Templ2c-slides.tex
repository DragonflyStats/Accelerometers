
%===================================================================================================%
\subsection*{2.6. Measuring the Global Risk}

Sections 2.1 through 2.5 discuss the theory of individual risks and the extension
of this approach to clusters such as households. ln many applications, however,
estimating a measure of global risk is preferred. Any global risk measure is result
in one single number that can be used to assess the risk of an entire micro dataset.
The following global risk measures are available in sdcMicroGUI, except the last
one presented in Section 2.7.2 that is computationally expensive is only made
available in sdcMicro.
%=================================================================%
\subsection*{2.6.1. Measuring the global risk using individual risks}
Two approaches can be used to determine the global risk for a dataset using
individual risks:
\begin{description}
	\item[Benchmark:] This approach counts the number of observations that can be con-
	sidered risky and also have higher risk as the main part of the data. For
	example, we consider units with individual risks being both $\geq 0.1$ and twice
	as large as the median of all individual risks + 2 times the median abso-
	lute deviation (MAD) of all unit risks. This statistics in also shown in the
	\textbf{sdcMicroGUI}.
	
	\item[Global risk:] The sum of the individual risks in the dataset gives the expected
	number of re—identifications
	
	%%%%CITE  [see Hundepool et al., 2008].
\end{description}
The benchmark approach indicates whether the distribution of individual risk
occurrences contains extreme values; it is a relative measure that depends on the
distribution of individual risks. It is not valid to conclude that observations with
higher risk as this benchmark are of very high risk; it evaluates whether some
unit risks behave differently compared to most of the other individual risks. The
global risk approach is based on an absolute measure of risk. 

Following is the print
output of the corresponding function from sdclvlicro, which shows both measures
(see the example in the manual of sdcMicro [Tcinpl ct al., 2013]):

%% CODE OUTPUT HERE

The global risk measurement taking into account this hierarchical structure if a
variable expressing it is defined.
%%-----------------------------------------------------------------------------------------------------%
%\subsection*{2.6.2. Measuring the global risk using log-linear models}
%Sample frequencies, considered for each of [W patterns 'm, fm ,m : 1, ..., 1% can
%be modeled by a Poisson distribution. In this case, global risk can be defined as
%the following [see also Skinner and Holmes, 1998]:
%
%%% EQUATION
%
%For simplicity, the (first order) inclusion probabilities are assumed to be equal,
%71',” = 7T ,m = 1. ..., ZVI. T1 can be estimated by log-linear models that include both
%the primary effects and possible interactions. This model is defined as:
%
%%% EQUATION
%
%To estimate the ,a,,,’s, the regression coefficients /2’ have to be estimated using,
%for example, iterative proportional fitting. The quality of this risk measurement
%approach depends o11 the number of different keys that result from cross-tabulating
%all key variables. If the cross—tabulated key variables are sparse in terms of how
%many observations have the same patterns, predicted values might be of low quality.
%It must also be considered that if the model for prediction is weak, the quality of
%the prediction of the frequency counts is also weak. Thus, the risk measurement
%%% Page 11 / 31
%%% 2 MEASURING THE DISCLOSURE RISK
%with log-linear models may lead to acceptable estimates of global risk only if not
%too many key variables are selected and if good predictors are available in the
%dataset.
%%----------------------------------------------------------------------------------------%
%In sdcMicro, global risk measurement using log-linear models can be completed
%with function \texttt{LLmodGlobalRisk()}. This function is experimental and needs further
%testing, however. It should be used only by expert users.

\subsection*{2.7. Measuring Risk for Continuous Key Variables}
The concepts of uniqueness and k—anonymity cannot be directly applied to con-
tinuous key variables because almost every unit in the dataset will be identified as
unique. As a result, this approach will fail. The following sections present methods
to measure risk for continuous key variables.


\subsection*{2.7.1. Distance-based record linkage}
If detailed information about a value of a continuous variable is available, i.e. the
risk comes from the fact that multiple datasets can be available to the attacker,
one of which contains identifiers like income, for example, attackers may be able
to identify and eventually obtain further information about an individual. Thus,
an intruder may identify statistical units by applying, for example, linking or
matching algorithms. The anonymization of continuous key variables should avoid
the possibility of successfully merging the underlying microdata with other external
data sources.
%-------------------------------------------------------------------------------------------------%
We assume that an intruder has information about a statistical unit included
in the microdata; the intruder’s information overlaps on some variables with the
information in the data. In simpler terms, We assume that the intruderls informa-
tion can be merged with niicrodata that should be secured. In addition, we also
assume that the intruder is sure that the link to the data is correct, except for
micro-aggregated data (see Section 3.4). Domingo-Fcrrcr and Torra [2001] showed
that these methods outperform probabilistic methods.

Mateo-Sanz et al. [2004] introduced distance-based record linkage and interval
disclosure. In the first approach, they look for the nearest neighbor from each
observation of the masked data value to the original data points. Then they mark
those units for which the nearest neighbor is the corresponding original value.

In the second approach, they check if the original value falls within an interval
centered on the masked value. Then they calculate the length of the intervals
based on the standard deviation of the variable under consideration (see Figure 2,
upper left graphic; the boxes expresses the intervals).
%--------------------------------------------------------------------------------------------------------------%
\subsection*{2.7.2. Special treatment of outliers when calculating disclosure risks}

It is worth to show alternatives to the previous distance-based risk measure. Such
alternatives took either distances between every observation into account or are
based on covariance estimation (as shown here). Thus, they are computationlly
more intensive, which is also the reason why they are not available in sdcMicroGUI
but only in sdcMicro for experienced users.

Almost all datasets used in official statistics contain units whose values in at least
one variable are quite different from the general observations. As a result, these
variables are very asymmetrically distributed. Examples of such outliers might
%%Page 12 / 31

%======================================================================================%
%% 2 MEASURING THE DISCLOSURE RISK
be enterprises with a very high value for turnover or persons with extremely high
income. In addition, multivariate outliers exist [see \textit{Templ and Meindl, 2008a}].

\begin{itemize}
	\item Unfortunately, intruders may want to disclose a large enterprise or an enterprise
	with specific characteristics. Since enterprises are often sampled with certainty or
	have a sampling weight close to 1, intruders can often be very confident that the
	enterprise they want to disclose has been sampled. 
	\item In contrast, an intruder may
	not be as interested to disclose statistical units that exhibit the same behavior as
	most other observations. For these reasons, it is good practice to define measures
	of disclosure risk that take the outlyingness of an observation into account. 
	% For details, see Templ and l\Iein<ll [2U08a]. 
	\item Outliers should be much more perturbed than
	non—outliers because these units are easier to re-identify even when the distance
	from the masked observation to its original observation is relatively large.
\end{itemize}

This method for risk estimation (called RMDID2 in Figure 2) is also included
in the \textbf{sdcMicro} package. 
It works as described in \textit{Templ and Meindl [2008a]} and
is listed as follows:
\begin{itemize}
	\item[(1.)] Robust mahalanobis distances (RMD) [see, for example Maronna ct al., 2006]
	are estimated between observations (continuous variables) to obtain a robust,
	multivariate distance for each unit.
	\item[(2.)] Intervals are estimated for each observation around every data point of the
	original data points. The length of the intervals depends on squared distances
	calculated in step 1 and an additional scale parameter. The higher the RMD
	of an observation, the larger the corresponding intervals.
	\item[(3.)] Check whether the corresponding masked values of a unit fall into the in-
	tervals around the original values. If the masked value lies within such an
	interval, the entire observation is considered unsafe. We obtain a vector in-
	dicating which observations are safe or which are not. For all unsafe units,
	at least m other observations from the masked data should be very close.
\end{itemize}
%--------------------------------------------------------------------------------------------------------%
Close is quantified by specifying a parameter for the length of the intervals
around this observation using Euclidean distances. If more than m points lie
within these small intervals, we can conclude that the observation is safe.
%--------------------------------------------------------------------------------------------------------%
Figure 2 depicts the idea of weighting disclosure risk intervals. For simple meth-
ods (top left and right graphics), the rectangular regions around each value are
the same size for each observation. Our proposed methods take the RMDs of
each observation into account. The difference between the bottom right and left
graphics is that, for method RMDID2, rectangular regions are calculated around
each masked variable as well. If an observation of the masked variable falls into an
interval around the original value, check whether this observation has close neigh-
bors. If the values of at least m other masked observations can be found inside a
second interval around this masked observation, these observations are considered
safe.


These methods are also implemented and available in sdcMicro as \texttt{dRisk()} and
\texttt{dRiskRMD()}. The former is automatically applied to objects of class sdcMi-
cr'0Obj, while the latter has to be specified explicitly and can currently not be
applied using the graphical user interface.
\newpage
Page 13 / 31

\textit{Figure 2: Original and corresponding masked observations (perturbed by adding
	additive noise). In the bottom right graphic, small additional regions are
	plotted around the masked values for RMDID2 procedures. The larger
	the intervals the more the observations is an outlier for the latter two
	methods.}
%====================================================%
\end{frame}
\end{document}