Statistical Disclosure Control for Survey Data
Chris Skinner
University of Southampton
Abstract
Statistical disclosure control refers to the methodology used in the design of the statistical
outputs from a survey for protecting the confidentiality of respondents’ answers. The
threat to confidentiality is assumed to come from a hypothetical intruder who has access
to these outputs and seeks to use them to disclose information about a survey respondent.
One key concern relates to identity disclosure, which would occur if the intruder were
able to link a known individual (or other unit) to an element of the output. Another main
concern relates to attribute disclosure, which would occur if the intruder could determine
the value of some survey variable for an identified individual (or other unit) using the
statistical output. Measures of the probability of disclosure are called disclosure risk. If
this level of risk is deemed unacceptable then it may be necessary to apply a method of
statistical disclosure control to the output. The choice of which method and how much
protection to apply depends not just on the impact on disclosure risk but also on the
impact on the utility of the output to users. This paper provides a review of statistical
disclosure control methodology for two main types of survey output: (i) tables of
estimates of population parameters and (ii) microdata, often released as a rectangular file
of variables by analysis units. For each of these types of output, the definition and
estimation of disclosure risk is discussed as well as methods for statistical disclosure
control.
%==================================================%
\subsection*{1. Introduction}
\subsection*{1.1. The problem of statistical disclosure control}
Survey respondents are usually provided with an assurance that their responses will
be treated confidentially. These assurances may relate to the way their responses will be
handled within the agency conducted the survey or they may relate to the nature of the
statistical outputs of the survey as, for example, in the ‘confidentiality guarantee’ in the
United Kingdom (UK) National Statistics Code of Practice (National Statistics, 2004,
p.7) that ‘no statistics will be produced that are likely to identify an individual’. This
paper is concerned with methods for ensuring that the latter kinds of assurances are met.

Thus, in the context of this paper, statistical disclosure control (SDC) refers to the
methodology employed, in the design of the statistical outputs from the survey, for
protecting the confidentiality of respondents’ answers. Methods relating to the first kind
of assurance, for example computer security and staff protocols for the management of
data within the survey agency, fall outside the scope of this paper.



There are various kinds of statistical outputs from surveys. The most traditional are
tables of descriptive estimates, such as totals, means and proportions. The release of such
l



estimates from surveys of households and individuals have typically not been considered
to represent a major threat to confidentiality, in particular because of the protection
provided by sampling. 


Tabular outputs from the kinds of establishment surveys
conducted by government have, however, long been deemed risky, especially because of
the threat of disclosure of information about large businesses in cells of tables which are
sampled with a 100\% sampling fraction. SDC methods for such tables have a long history
and will be outlined in Section 2.

While the traditional model of delivering all the estimates from a survey in a single
report continues to meet certain needs, there has been increasing demand for more
flexible survey outputs, often for multiple users, where the set of population parameters
of interest is not pre-specified. There are several reasons why it may not be possible to
pre-specify all the parameters. 

Data analysis is an iterative process and what analyses are
of most interest may only become clear after initial exploratory analyses of the data.
Moreover, given the considerable expense of running surveys, it is natural for many
commissioners of surveys to seek to facilitate the use of the data by multiple users. But it
is usually impossible to pre-specify all possible users and their needs in advance. 

A
natural way to provide flexible outputs from a survey to address such needs is to make
the survey microdata available so that users can carry out the statistical analyses that
interest them.

The release of such microdata raises serious confidentiality protection issues,
however. Of course, statistical analyses of survey data do not require that the identities of
the survey units are known. Names, addresses and contact information for individuals or
establishment can be stripped from the data to form an anonymised microdata file. The
problem, however, is that such basic anonymisation is often insufficient to protect
confidentiality, and it is necessary therefore to employ one of a range of alternative
approaches to SDC and this will be discussed further in Section 3.

%----------------------------------------------------------------------------------%
\subsection*{1.2 Concepts of confidentiality, disclosure and disclosure risk}
To be precise about what is meant by ‘protecting confidentiality’ requires discussion
of definitions. These usually involve the notion of a hypothetical intruder who might seek
to breach confidentiality. 
There are thus three key parties: 
\begin{itemize}
\item[(i)] the respondent who provides the data, 
\item[(ii)] the agency which collects the data, releases statistical outputs and
designs the SDC strategy, and 
\item[(iii)] the hypothetical intruder who has access to these
outputs and seeks to use them to disclose information about the respondent.
\end{itemize}

 One
important notion of disclosure is identity disclosure or identification, which would occur
if the intruder linked a known individual (or other unit) to an individual microdata record
or other element of the statistical output. Another important notion is \textbf{\textit{attribute disclosure}},
which would occur if the intruder could determine the value of some survey variable for
an identified individual (or other unit) using the statistical output. 

More generally,
prediction disclosure would occur if the intruder could predict the value of some survey
variable for an identified individual with some uncertainty. When assessing the potential
for disclosure for a particular statistical output, it is usual to refer to the disclosure risk.

This might be defined as the probability of disclosure with respect to specified sources of
uncertainty. Or the term might be used loosely to emphasize not only the uncertainty
about potential disclosure but also the potential harm that might arise from disclosure
(Lambert, 1993). The confidentiality of the answers provided by a respondent might be
2



said to be protected if the disclosure risk for this respondent and the respondent’s answers
is sufficiently low. We shall discuss disclosure risk in more detail in sections 2 and 3.
For further discussion of definitions of disclosure see Duncan and Lambert (1986; 1989)
and Skinner (1992).
%-------------------------------------------------------------------------------%
\subsection*{1.3 Approaches to protecting confidentiality}
If the disclosure risk is not deemed to be sufficiently low, then it will be necessary to
employ some method to reduce the risk. There are broadly two approaches, which are
referred to here as sqfe setting and safe data (Marsh et al. 1994). 

The safe setting
approach imposes restrictions on the set of possible users of the statistical output and/or
on the ways that the output can be used. For example, users might be required to sign a
licencing agreement or might only be able to access microdata by visiting a secure
laboratory or by submitting requests remotely (National Research Council, 2005). The
safe data approach, on the other hand, involves some modification to the statistical
output. 

For example, the degree of geographical detail in a microdata file from a national
social survey might be limited so that no area containing less than 100,000 households is
identified. In this paper we focus on the safe data approach and generally refer to
methods for modifying the statistical output as SDC methods.
%-------------------------------------------------------------------------------%
\subsection*{1.4 SDC methods. utilitv and data quality}
SDC methods vary according to the form of the statistical output. Some simple
approaches are:
\begin{itemize}
\item reduction of detail, for example the number of categories of a categorical
variable might be reduced in a cross-classified table or in microdata;
\item suppression, for example the entry in a table might be replaced by an asterisk,
indicating that the entry has been suppressed for confidentiality reasons.
\end{itemize}
%----------------------------------------------------------------------------------------%

In each of these cases, the SDC method will lead to some loss of infarmation for the
user of the statistical output. Thus, the method will reduce the number of population
parameters for which a user can obtain survey estimates. Other kinds of SDC methods
might not affect the number of parameters which can be estimated but may affect the
quality of the estimates that can be produced. 

For example, if random noise is added to an
income variable to protect confidentiality then this may induce bias or variance inflation
in associated survey estimates. The general term utility may be used to cover both the
information provided by the statistical outputs, e.g. the range of estimates or analyses
which can be produced, and the quality of this information, e.g. the extent of errors in
these estimates. It should, of course, be recognized that survey data are subject to many
sources of error, even prior to the application of SDC methods, and the impact of SDC
methods on data quality therefore needs to be considered in this context.

Utility generally needs to be considered from the perspective of a user of the
statistical outputs, who represents a key fourth party to add to the three parties referred to
earlier: the respondent, the agency and the intruder.
%---------------------------------------------------------------------------------------%
\subsection*{1.5 SDC as an optimisation problem: the risk-utilitv trade-off}
The key challenge in SDC is how to deal with the trade-off between disclosure risk
and utility. In general, the more the disclosure risk is reduced by an SDC method, the
lower will be the expected utility of the output. This trade-off may be formulated as an
optimisation problem. 

Let D be the (anonymized) survey data and let f(D), be the
statistical output, resulting from the use of an SDC method. Let R[ f (D)] be a measure of
the disclosure risk of the output and let U [ f (D)] be a measure of the utility of the
output. Then the basic challenge of SDC might be represented as the constrained
optimisation problem :
for given D and 8, find an SDC method, $f (.)$, which:
maximises U[f(D)], subject to $R[f(D)] < 5$.

The elements of this problem need some clarification:
\begin{itemize}
\item[f(.)] : the SDC method - a wide variety of these have been proposed and we shall refer to
some of these in this paper;
\item R(.)] : the disclosure risk function - we shall discuss ways in which this function may be
defined; this is certainly not straightforward, e.g. because of its dependence on
assumptions about the intruder and because of the challenge of combining the
threats of disclosure for multiple respondents into a scalar function;
\item[U (.)] : the utility function - this will also not be straightforward to specify as a scalar
function, given the potential multiple uses of the output;
\end{itemize}
%----------------------------------------------------------------------------------%
8 maximum acceptable risk: in principle, one might expect the agency to provide this
value in the light of its assurances to respondents. However, in practice, agencies
find it very difficult to specify a value of 5, other than Zero, i.e. no disclosure risk.
Unfortunately, for most definitions of disclosure risk, the only way to achieve no
disclosure risk is by not releasing any output and this is rarely a solution of interest!
%------------------------------------------------------------------------------------------%
Given these difficulties in specifying R(.) and U (.) as scalar functions and in
specifying a value for E, the above optimization problem serves mainly as conceptual
motivation. In practice, different SDC methods can be evaluated and compared by
considering the values of alternative measures of risk and utility. For given measures of
each, it can sometimes be useful to construct an RU map (Duncan et al., 2001), where a
measure of risk is plotted against a measure of utility for a set of candidate SDC methods.

The points on this map are expected to display a general positive relationship beween risk
and utility, but one might still find that, for given values of risk, some methods have
greater utility than others and thus are to be preferred. This approach avoids having to
assume a single value of 6.
%=====================================================%
\newpage
\section*{2. Tabular Outputs}
\subsection*{2.1 Disclosure risk in social survevs and the protection provided bv sampling}
The main developments in SDC methods for tabular outputs have been motivated by
the potential risks of disclosure arising when 100\% sampling has been employed, such as
in censuses or in administrative data. Frequency tables based upon such data sources
may often include small counts, as low as Zero or one, e.g. in tables of numbers of deaths
by area by cause of death. Such tables might lead to identity disclosure, e.g. if it is public
knowledge that someone has died, then it might be possible to identify that person as a
count of one in a table of deaths using some known characteristics of that person.
%-------------------------------------------------------------%

Attribute disclosure might also occur. For example, it might be possible to find out the
cause of the person’s death if the table cross-classifies this cause by other variables
potentially known to an intruder.
\begin{itemize}
\item In social surveys, however, the use of sampling greatly reduces the risks of such kinds
of disclosure for two reasons. Firstly, the presence of sampling requires different kinds of
statistical outputs. Thus, the entries in tables for categorical variables tend to be weighted
proportions (possibly within domains defined by rows or columns) and not unweighted
sample counts. 
\item Even if a user of the table could work out the cell counts (e. g. because the
survey employs equal weights and the sample base has been provided), the survey agency
will often ensure that the published cells do not contain very small counts, where the
estimates would be deemed too unreliable due to sampling error. 
\item For example, the agency
might suppress cell entries where the sample count in the cell falls below some threshold,
e.g. 50 persons in a national social survey. This should prevent the kinds of situations of
most concem with 100\% data. Sometimes, agencies use techniques of small area
estimation in domains with small sample counts and these techniques may also act to
reduce disclosure risk.
%---------------------------------------------------------%
\item Secondly, the presence of sampling should reduce the precision with which an
intruder could achieve predictive disclosure. For example, suppose that an intruder could
find out from a survey table that, among 100 respondents falling into a certain domain, 99
of them have a certain attribute and suppose that the intruder knows someone in the
population who falls into this domain. 
\item Then the intruder cannot predict that this person
has the attribute with probability 0.99, since this person need not be a respondent and
prediction is subject to sampling uncertainty. This conclusion depends, however, on the
identities of the survey respondents being kept confidential by the agency, preventing the
intruder knowing whether the known person is a respondent, referred to as \textbf{response
knowledge} by Bethlehem et al. (1980). 

\item In general, it seems very important that agencies
do adopt this practice since it greatly reduces disclosure risk, while not affecting the
statistical utility of the outputs. In some exceptional cases, it may be difficult to achieve
this completely. For example, in a survey of children it will usually be necessary to
obtain the consent of a child’s parent (or other adult) in order for the child to take part in
the survey. The child might be assured that their responses will be kept confidential from
their parent. However, when examining the outputs of the survey, the parent (as intruder)
would know that their child was a respondent.
\end{itemize}

%---------------------------------------------------------%
For the reasons given above, disclosure will not generally be of concern in the release
of tables of estimates from social surveys, where the sample inclusion probabilities are
small (say never exceeding 0.1). See also Federal Committee on Statistical Methodology
(2005, pp. l2-14).

\subsection*{2.2. Disclosure risk in establishment survevs}
A common form of output from an establishment survey consists of a table of
estimated totals, cross-classified by characteristics of the establishment. Each estimate
takes the form  =ZSw,I‘_,y, , where w, is the survey weight, IL, is a 0-1 indicator for
cell c in the cross-classification and y, is the survey variable for the in‘ establishment in
the samples. For example, y, might be a measure of output and the cells might be
formed by cross-classifying industrial activity and a measure of size.
5


\subsection*{Prediction disclosure}
The relevant definition of disclosure in such a setting will often be a form of
prediction disclosure. Prediction disclosure for a specific cell c might be defined under
the following set-up and assumptions:
\begin{itemize}
\item the intruder is one of the establishments in the cell which has the aim of predicting
the value y, for one of the other establishments in the cell or, more generally, the
intruder consists of a coalition of m of the N‘ establishments in the cell with the same
predictive aim;
\item the intruder knows the identities of all establishments within the cell (since, for
example, they might represent businesses competing in a similar market).
\end{itemize}

Given such assumptions, prediction disclosure might be said to occur if the intruder is
able to predict the value yt with a specified degree of precision. In order to clarify the
notion of precision, we focus in the next subsection on the important case where the units
in the cell all fall within completely enumerated strata. Thus, w, =1 when I“ =1 so that
la, =21] y, , where UL is the set of all establishments in cell c and Np is the size of UL .
In this case the intruder faces no uncertainty due to sampling and this might therefore be
treated as the worst case.
%----------------------------------------------------------------------------------------------------%
\subsection*{2.2.1 Prediction disclosure in the absence of sampling}
In the absence of sampling, prediction is normally considered from a deterministic
perspective and is represented by an interval (between an upper and lower bound) within
which the intruder knows that a value y, must lie. The precision of prediction is
represented by the difference between the true value and one of the bounds. It is supposed
that the intruder undertakes prediction by combining prior information with the reported
value Y6.
One approach to specifying the prior information is used in the prior-posterior rule
(Willenborg and de Waal, 2001), also called the pq rule, which depends upon two
constants, p and q, set by the agency. The constant q is used to specify the precision of
prediction based upon the prior information alone. Under the pq rule, it is assumed that
intruder can infer the yl value for each establishment in the cell to within q%. Thus, the
agency assumes that, prior to the table being published, the intruder could know that a
value y. falls within the interval [(1—q/l0O)y,,(l+q/l00)y,]. The combination of this
t
prior information with the output IZU yl can then be used by the intruder to obtain
sharper bounds on a true value. For example, let ym syw $...$ ym) be the order
statistics and suppose that the intruder is the establishment with the second largest value,
ym _,). Then this intruder can determine an upper bound for the largest value yw) by
subtracting its own value ym, _1, together with the sum of the lower bounds for
ym,..., y(N _2) from  The precision of prediction using this upper bound is given by
difference between this upper bound and the true value yum, which is (q/100)2Z] 2 ya) .
This cell would be called sensitive under the pq rule, i.e. judged disclosive, if this
difference was less than 12% of the true value, i.e. if
6



(p/lO0)y(N()—(q/lOO)ZZl’2y(,) >0. <1)
The expression on the left hand side of (l) is a special case of a linear sensitivity
measure, which more generally takes the form R‘ =2: a,yW , where the a, are specified
weights. The cell is said to be sensitive if R‘. >0. In this case, prediction disclosure
would be deemed to occur. A widely used special case of the pq rule is the 12% rule,
which arises from setting q=100, i.e. no prior information is assumed. Another
commonly used linear sensitivity measure arises with the (n,k) or dominance rule. See
Willenborg and de Waal (2001), Cox (2001), Giessing (2001) and Federal Committee on
Statistical Methodology (2005) for further discussion.
%------------------------------------------------------------------------------------------------%
\subsection*{2.2.2 Prediction disclosure in the presence of sampling}
More generally, all cell units may not be completely enumerated. In this case,  will
be subject to sampling error and, in general, this will lead to additional disclosure
protection, provided the intruder does not know whether other establishments (other than
those in the coalition) are sampled or not. 

The definition of risk in this setting appears to
need further research. Willenborg and de Waal (2001, section 6.2.5) present some ideas.
An alternative model-based stochastic approach might assume that before the release of
the table, the prior information about the y, can be represented by a linear regression
model depending upon publicly available covariate values x, with a specified residual
variance. 

The predictive distribution of y, given xi could then be updated using the
known value(s) of yl. for the intruder and the reported 2., which might be assumed to
follow the distribution $Yr \sim \mathcal{N} [Y(,v(l?()],$ where v(l;(.) is the reported variance estimate of
la. Prediction disclosure could then be measured in terms of the resulting residual
variance in the prediction of yl.
%-------------------------------------------------------------------------------------------------------%
\subsection*{2.3. SDC methods for tabular outputs}
If a cell in a table is deemed sensitive, i.e. the cell value represents an unacceptably high
disclosure risk, a number of SDC approaches may be used.
\begin{description}
\item[Redefinition of cells:] The cells are redefined to remove sensitive cells, e.g. by combining
sensitive cells with other cells or by combining categories of the cross-classified
variables. This is also called table redesign (Willenborg and de Waal, 2001).
\item[Cell suppression:] the value of a sensitive cell is suppressed. Depending upon the nature
of the table and its published margins, it may also be necessary to suppress the values of
‘complementary’ cells to prevent an intruder being able to deduce the value of the cell
from other values in the table. 
\end{description}
%---------------------------------------------------------------------------------------------------------%
There is a large literature on approaches to choosing
complementary cells which ensure disclosure protection. See e.g. Willenborg and de
Waal (2001), Cox (2001) and Giessing (2001) and references therein.
Cell modification: the cell values may be modified in some way. It will generally be
necessary to modify not only the values in the sensitive cells but also values in some
complementary non-sensitive cells, for the same reason as in cell suppression.
Modification may be detenninistic, e.g. Cox et al. (2004), or stochastic, e.g. Willenborg
7
