\documentclass[SDCmaster.tex]{subfiles}
\begin{document}
	\begin{frame}
\frametitle{k-anonymity}
\begin{itemize}
\item \textbf{k-anonymity} is a property possessed by certain anonymized data. 
\item The concept of k-anonymity was first formulated by Latanya Sweeney in a paper published in 2002 as an attempt to solve the problem: 
\begin{quote}"Given person-specific field-structured data, produce a release of the data with scientific guarantees that the individuals who are the subjects of the data cannot be re-identified while the data remain practically useful."
\end{quote}
\end{itemize}
\end{frame}
	\begin{frame}
		\frametitle{k-anonymity}
		\begin{itemize}
\item A release of data is said to have the k-anonymity property if the information for each person contained in the release cannot be distinguished from at least k-1 individuals whose information also appear in the release. 
\item The various procedures and programs for generating anonymised data providing k-anonymity protection have been patented in the United States.
\end{itemize}
\end{frame}
\end{document}
%===========================================================%