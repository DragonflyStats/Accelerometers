\documentclass[]{article}
\usepackage{framed}

\begin{document}



\section{Data Analysis : Week 2 Quiz}


\subsection*{Question 1}

In the text of the final write-up of a data analysis, how should the analyses be reported?

\begin{itemize} 
\item[(i)] Every analysis performed should be reported with a measure of uncertainty. (Not all data analyses contain an element of "uncertainty"- censuses)
\item[(ii)] Analyses should be reported in the order that they appear in the raw scripts files.
\item[(iii)] Analyses should be reported in an order to convey the story being told with the data analysis.
\item[(iv)] Analyses should be reported in chronological order of when they are performed. (useful but not a major concern)
\end{itemize}

%**Analyses should be reported in an order to convey the story being told with the data analysis.**

%----------------------------------------------------------------%
\newpage
\subsection*{Question 2}

\begin{itemize}
\item Open a connection to the old version of my blog: `http://simplystatistics.tumblr.com/`, 
\item read the first 150 lines of the file and assign them to a vector `simplyStats`. 
\item Apply the `\texttt{nchar()}` function to `simplyStats` to count the characters in each element of `simplyStats`. 
\item How many characters long are the lines 2, 45, and 122?
\end{itemize}


\begin{framed}
\begin{verbatim}

# open a connection to http://simplystatistics.tumblr.com/ 
# assign to vector `simplyStats`
simplyStats <- readLines( 
 url('http://simplystatistics.tumblr.com/'), 150)
\end{verbatim}
\end{framed}

How many characters are in each element of `simplyStats`. 
\begin{framed}
\begin{verbatim}
# apply `nchar()`
simplyStatsChars <- nchar(simplyStats)
\end{verbatim}
\end{framed}

How many characters long are the lines 2, 45, and 122?
\begin{framed}
\begin{verbatim}
# how many characters long is line 2?
nchar(simplyStats)[2]

# how many characters long is line 45?
nchar(simplyStats)[45]

# how many characters long is line 122?
nchar(simplyStats)[122]
\end{verbatim}
\end{framed}

%-----------------------------------------------------------------%
\newpage
\subsection*{Question 3}

The American Community Survey distributes downloadable data about United States communities. 
Download the 2006 microdata survey about housing for the state of Idaho using \texttt{download.file()} from here: 

\begin{verbatim}
https://dl.dropbox.com/u/7710864/data/csv_hid/ss06hid.csv

or here:

https://spark-public.s3.amazonaws.com/dataanalysis/ss06hid.csv 
\end{verbatim}
and load the data into \texttt{R}. You will use this data for the next several questions. 

\noindent \textbf{\textit{Code Book}}\\
The code book, describing the variable names is here: 

\begin{verbatim}
https://dl.dropbox.com/u/7710864/data/PUMSDataDict06.pdf

or here: 

https://spark-public.s3.amazonaws.com/dataanalysis/PUMSDataDict06.pdf
\end{verbatim}
\bigskip
How many housing units in this survey were worth more than \$1,000,000?
% **53**

\begin{framed}
\begin{verbatim}
# Download 2006 microdata survey 
# re: housing for Idaho using download.file()
# setwd("~/DA")
download.file(
 'https://spark-public.s3.amazonaws.com/dataanalysis/ss06hid.csv',
              "ss06hid.csv", method="curl")

# Download the code book:
# download.file(
 'https://spark-public.s3.amazonaws.com/dataanalysis/PUMSDataDict06.pdf',
              "PUMSDataDict06.pdf", method="curl")

# load the data into R
idahoData <- read.csv("ss06hid.csv", header=TRUE)

# are we sure it's just Idaho data?
table(idahoData$ST)
#Check the PDF - what does 16 mean?

#any missing data?
summary(idahoData$ST)

# How many housing units [are] worth more than $1,000,000?
table(idahoData$TYPE,idahoData$VAL)
\end{verbatim}
\end{framed}

\begin{framed}
\begin{verbatim}
#from local files
idahoData <- read.csv("daquiz2.csv", header=TRUE)

\end{verbatim}
\end{framed}

%-----------------------------------------------------------------%
\newpage
\subsection*{ Question 4}

\begin{itemize}
\item Use the data you loaded from Question 3. 
\item Consider the variable FES. 
\item Which of the "tidy data" principles does this variable violate?
\end{itemize}

%READY
\textbf{\textit{Revision}}\\
What are the three characteristics of tidy data?

\begin{itemize}
\item ``\textit{\textbf{Tidy data}}" by Hadley Wickham (RStudio)
\item Submission to Journal of Statistical Software
\item (http://vita.had.co.nz/papers/tidy-data.pdf)
\end{itemize}
Three Principles from Hadley Wickham's paper
\begin{itemize}
\item[1.] Each variable forms a column, 
\item[2.] Each observation forms a row, 
\item[3.] Each table/file stores data about one kind of observation.
\end{itemize}

\begin{framed} 
\begin{verbatim}
# let's look!
unique(idahoData$FES)
\end{verbatim}
\end{framed} 
\textbf{Options}
\begin{itemize}
\item[(i)]  Each tidy data table contains information about only one type of observation.\\
(Not so)

\item[(ii)]  Each variable in a tidy data set has been transformed to be interpretable.
(No)

\item[(iii)]  Tidy data has no missing values.

\item[(iv)]  Tidy data has one variable per column.
\end{itemize}
  
%-----------------------------------------------------------------%
\newpage
\subsection*{ Question 5 }

Use the data you loaded from Question 3. 

\begin{itemize}
\item How many households have 3 bedrooms and and 4 total rooms? 
\item How many households have 2 bedrooms and 5 total rooms? 
\item How many households have 2 bedrooms and 7 total rooms?
\end{itemize}
\begin{framed}
\begin{verbatim}
#USING TABLE
#Rooms on Rows , Bedrooms on Columns
#dnn adds dimension names

table(idahoData$RMS,idahoData$BDS,dnn=list("RMS","BDS"))

\end{verbatim}
\end{framed}
Another Way of Doing it
\begin{framed}
\begin{verbatim}
# How many households have 3 bedrooms and 4 total rooms?
nrow(idahoData[!is.na(idahoData$BDS) & idahoData$BDS==3 &
                 !is.na(idahoData$BDS) & idahoData$RMS==4,])
# How many households have 2 bedrooms and 5 total rooms?
nrow(idahoData[!is.na(idahoData$BDS) & idahoData$BDS==2 &
                 !is.na(idahoData$BDS) & idahoData$RMS==5,])
# How many households have 2 bedrooms and 7 total rooms?
nrow(idahoData[!is.na(idahoData$BDS) & idahoData$BDS==2 &
                 !is.na(idahoData$BDS) & idahoData$RMS==7,])

\end{verbatim}
\end{framed}
% **148, 386, 49**


%-----------------------------------------------------------------%
\newpage
\subsection*{Question 6}
\begin{itemize}
\item Use the data from Question 3. 
\item Create a logical vector that identifies the households on greater than 10 acres who sold more than \$10,000 worth of agriculture products. 
\item Assign that logical vector to the variable `\texttt{agricultureLogical}`. 
\item Apply the `\texttt{which()} function like this to identify the rows of the data frame where the logical vector is `TRUE`.
\end{itemize}

\begin{framed} 
\begin{verbatim}
# Like this (this wont run yet)
 which(agricultureLogical) 
\end{verbatim}
\end{framed} 

What are the first 3 values that result?

\begin{framed} \begin{verbatim}
# Showing off a bit
q6cols <- c("ACR", "AGS")
which(names(idahoData) %in% q6cols)  

# logical vector
agricultureLogical <- idahoData$ACR==3 & idahoData$AGS==6

# and:
 which(agricultureLogical) 
\end{verbatim}\end{framed} 

%**125, 238, 262**

%-----------------------------------------------------------------%
\newpage
\subsection{Question 7}

\begin{itemize}
\item Use the data from Question 3. 
\item Create a logical vector that identifies the households on greater than 10 acres who
 sold more than \$10,000 worth of agriculture products. 
\item Assign that logical vector to the variable \texttt{agricultureLogical}. 
\item Apply the \texttt{which()} function like this to identify the rows of the 
data frame where the logical vector is TRUE and assign it to the variable indexes. 
\end{itemize}

\begin{framed} \begin{verbatim}
indexes =  which(agricultureLogical) 
\end{verbatim}\end{framed} 

If your data frame for the complete data is called \texttt{dataFrame} you can create a data frame 
with only the above subset with the command: 

\begin{framed} 
\begin{verbatim}
subsetDataFrame  = dataFrame[indexes,] 
\end{verbatim}
\end{framed} 

\noindent Note that we are subsetting this way because the NA values in the variables 
will cause problems if you subset directly with the logical statement. 


\noindent How many households in the subsetDataFrame have a missing value for the mortgage status 
(MRGX) variable?

\begin{framed} 
\begin{verbatim}
indexes <- which(agricultureLogical)
subsetIdahoData <- idahoData[indexes,]

# And then:
nrow(subsetIdahoData[is.na(subsetIdahoData$MRGX),])
\end{verbatim}
\end{framed} 

%**8**
%-----------------------------------------------------------------%
\newpage
\subsection*{Question 8}
\begin{itemize}
\item Use the data from Question 3.
\item Apply `\texttt{strsplit()}` to split all the names of the data frame on the characters "wgtp". 
\item What is the value of the 123 element of the resulting list?
\end{itemize}

\begin{framed} \begin{verbatim}
List <- strsplit(names(idahoData), "wgtp")
List[123]
\end{verbatim}\end{framed} 

%**"" "15"**

%-----------------------------------------------------------------%
\newpage
\subsection*{Question 9}

What are the 0\% and 100\% quantiles of the variable \texttt{YBL}? Is there anything wrong with these values?
\textit{ Hint: you may need to use the \texttt{na.rm} parameter.}

\begin{framed} 
\begin{verbatim}
quantile(idahoData$YBL, na.rm=TRUE)
#  0%  25%  50%  75% 100% 
#  -1    3    5    7   25 
\end{verbatim}
\end{framed} 

%-----------------------------------------------------------------%
\newpage
\subsection*{Question 10}

In addition to the data from Question 3, the American Community Survey also collects data about populations. 
Using `download.file()`, download the population record data from: 

\begin{verbatim}
https://dl.dropbox.com/u/7710864/data/csv_hid/ss06pid.csv 

or here:

https://spark-public.s3.amazonaws.com/dataanalysis/ss06pid.csv
\end{verbatim}

\begin{itemize}
\item Load the data into \texttt{R}. Assign the housing data from Question 3 to a data frame `\texttt{housingData}` and the population data from above to a data frame `populationData`.

\item Use the merge command to merge these data sets based only on the common identifier "SERIALNO". 

\item What is the dimension of the resulting data set? 
\end{itemize}
%[OPTIONAL] For fun, you might look at the data and see what happened when they merged.

\begin{framed} 
\begin{verbatim}
download.file(
'https://spark-public.s3.amazonaws.com/dataanalysis/ss06pid.csv',
              'ss06pid.csv', method='curl')

rm(idahoData)
housingData <- read.csv("ss06hid.csv", header=TRUE)
populationData <- read.csv("ss06pid.csv", header=TRUE)

dim(merge(housingData, 
 populationData, by="SERIALNO", all=TRUE))
\end{verbatim}
\end{framed} 

% **number of rows = 15451, number of columns = 426**

\end{document}
\end{document}
