\documentclass{beamer}

\usepackage{amsmath}
\usepackage{framed}
\usepackage{amssymb}

\begin{document}

%---------------------------------------------%
\begin{frame}
\frametitle{Deploying Shiny apps}
\Large
\vspace{-1cm}
\begin{itemize}
\item  The Shiny package itself is designed to run Shiny applications locally. 
\item To share Shiny applications with other R users, you can send them your application source as a GitHub gist, R package, or zip file.
\end{itemize}
\end{frame}

\begin{frame}
\frametitle{Deploying Shiny}
\Large
\textbf{Sharing Apps to Run Locally}
\begin{itemize}
\item Once you’ve written your Shiny app, you can distribute it for others to run on their own computers—they can download and run Shiny apps with a single \texttt{R} command.
\item  All that this requires that they have \texttt{R} and Shiny installed on their computers.


\end{itemize}
%Here are some ways to deliver Shiny apps to run locally:

\end{frame}
%-----------------------------------------------------------------------------------%
\begin{frame}
\frametitle{Deploying Shiny}
\Large
\textbf{Gist}
\begin{itemize}
\item One easy way is to put your code on \texttt{gist.github.com}, a code pasteboard service from \textbf{GitHub}. 
\item Both \texttt{server.R} and \texttt{ui.R} must be included in the same gist, and you must use their proper filenames. 
\item See \textit{http://gist.github.com/3239667} for an example.
\end{itemize}
\end{frame}
%-------------------------------------------------------------------------%
\begin{frame}[fragile]
\Large
\begin{itemize}
\item Your recipient must have R and the Shiny package installed, and then running the app is as easy as entering the following command:

\begin{framed}
\begin{verbatim}
shiny::runGist('3239667')
\end{verbatim}
\end{framed}
\item In place of '\textit{\textbf{3239667}}' you will use your gist’s ID; or, you can use the entire URL of the gist (e.g. '\textit{https://gist.github.com/3239667}').
\end{itemize}
\end{frame}

\begin{frame}
\Large
\frametitle{Deploying Shiny}
\textbf{Advantages of using Gist} 
\begin{itemize}
\item Source code is easily visible by recipient (if desired)
\item Easy to run (for \texttt{R} users)
\item Easy to post and update
\end{itemize} 
\textbf{Cons} \begin{itemize}
\item Code is published to a third-party server
\end{itemize}
\end{frame}

%-----------------------------------------------------------------------------------%
\begin{frame}[fragile]
\Large
\textbf{GitHub repository}
\begin{itemize}
\item If your project is stored in a git repository on GitHub, then others can download and run your app directly. An example repository is at \textit{\textbf{http://github.com/rstudio}}
\item The following command will download and run the application:
\end{itemize}
\large
\begin{framed}
\begin{verbatim}
shiny::runGitHub(`shiny_example', `rstudio')
\end{verbatim}
\end{framed}
\normalsize
\textit{In this example, the GitHub account is 'rstudio' and the repository is 'shiny example'; you will need to replace them with your account and repository name.}
\end{frame}

\begin{frame}
\Large
\textbf{Github: Advantages} \begin{itemize}
\item  Source code is easily visible by recipient (if desired)
\item Easy to run (for R users)
\item Very easy to update if you already use GitHub for your project
\item Git-savvy users can clone and fork your repository
\end{itemize} \textbf{Disadvantages} \begin{itemize}
\item Developer must know how to use git and GitHub.
\item Code is hosted by a third-party server.
\end{itemize}
\end{frame}


%-----------------------------------------------------------------------------------%
\begin{frame}[fragile]
\frametitle{Deploying Shiny}
\textbf{Making it into a Package}
\begin{itemize}
\item If your Shiny app is useful to a broader audience, it might be worth the effort to turn it into an \texttt{R} package. Put your Shiny application directory under the package’s inst directory, then create and export a function that contains something like this:
\begin{framed}
\begin{verbatim}
shiny::runApp(system.file('appdir', 
 package='packagename'))
\end{verbatim}
\end{framed}
where \texttt{appdir} is the name of your app’s subdirectory in inst, and \textbf{\emph{packagename}} is the name of your package.
\end{itemize}
\end{frame}

\begin{frame}
\frametitle{Deploying Shiny}
\textbf{Making it into a Package}:\\ \bigskip
\textbf{Advantages} \begin{itemize}
\item Publishable on CRAN
\item Easy to run (for \texttt{R} users)
\end{itemize} \textbf{Disadvantages} \begin{itemize}
\item More work to set up
\item Source code is visible by recipient (if not desired)
\end{itemize}
\end{frame}

%---------------------------------------------%
\begin{frame}
\frametitle{Deploying Shiny apps}
\Large
\textbf{Deployment over the Web}
\begin{itemize}
\item You can also deploy Shiny applications over the web, so that users need only a web browser and your application’s URL.
\item For this, you’ll need a Linux server and our Shiny Server software. 
\item Shiny Server is free and open source, though in the future RStudio will offer a commercially licensed edition with additional features for larger organizations.
\item RStudio also working on a subscription-based hosting service for Shiny. 
\end{itemize}
\end{frame}
%---------------------------------------------%

\begin{frame}
\frametitle{Deploying Shiny apps : Shiny Server}
\Large
\textbf{Shiny Server}
\begin{itemize}


\item Shiny Server is if you want to use your own server instead of hosting it on Rstudio's server (i.e. \textbf{\textit{glimmer}}). 

\item  This is really important for those who can't let their code or data out of their organization, 
or want more computational/storage resources than glimmer can offer, or need their apps to access their 
internal network.
\end{itemize}
\end{frame}
%---------------------------------------------%

\end{document}